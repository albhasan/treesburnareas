\PassOptionsToPackage{table}{xcolor}
\documentclass[aspectratio=169]{beamer}

\setbeamertemplate{footline}[frame number]

% \usepackage{hyperref}
\usepackage{caption}

\usepackage{booktabs}
\usepackage{longtable}
\usepackage{array}
\usepackage{multirow}
\usepackage{wrapfig}
\usepackage{float}
\usepackage{pdflscape}
\usepackage{tabu}
\usepackage{threeparttable}
\usepackage{threeparttablex}
\usepackage[normalem]{ulem}
\usepackage{makecell}
\usepackage{colortbl}
\usepackage{xcolor}

\captionsetup[figure]{labelformat=empty}

\title{Exploratory analysis of Recurrent deforestation warnings in the 
Brazilian Amazon}

\author{Alber Sanchez\\alber.ipia@inpe.br}
\institute{
    \includegraphics[width=4cm,keepaspectratio]{./logos/trees-color-h_2.png}
    \includegraphics[width=1.8cm,keepaspectratio]{./logos/logoinpe-azul-menor.png} \\
    TreesLab\\Instituto Nacional de Pesquisas Espaciais - INPE\\Brazil
}
\date{\today}

\begin{document}

\frame{\titlepage}





\section{Introduction}

\begin{frame}
    \frametitle{Introduction}
    \begin{itemize}
        \item Deforestation by successive degradation remains a challenging 
            question in the scientific literature.
        \item We think an answer to this question lies down in DETER data.
        \item This answer could play an important role, for example, in the 
            brazilian estimation of greenhouse gases.
        \item We used DETER data from 2016 to 2021 of the Amazon Biome in 
              Brazil.
    \end{itemize}
\end{frame}



\subsection{DETER}

\begin{frame}
    \frametitle{What is DETER?}
    \begin{itemize}
        \item DETER is a GIS which produces a fast assessment of deforestation 
            and forest degradation in the Brazilian 
            Amazon~\cite{shimabukuro2006}.
        \item DETER is an important tool for environmental protection and  
            effective law enforcement.
        \item DETER employs Linear Mixture Models of CBERS imagery and human
            experts to deter and issue warnings of deforested (or degraded) 
            areas larger than 3 ha~\cite{dealmeida2022}.
        \item Annually, DETER takes from PRODES the current forested area, 
            stating anew issuing warnings.
    \end{itemize}
\end{frame}

\begin{frame}
    \frametitle{DETER warnings}
    \begin{figure}[h] 
        \includegraphics[width=0.75\linewidth]
        {./figures/plot_deter_area_by_state_pyear_type.png}
        \label{fig:deter_area_state_pyear_type}
    \end{figure}
\end{frame}

% \begin{frame}[allowframebreaks]
%     \frametitle{DETER warnings}
%     \begingroup\fontsize{7}{9}\selectfont

\begin{longtabu} to \linewidth {>{\raggedright}X>{\raggedright}X>{\raggedright}X>{\raggedleft}X}
\toprule
UF & year & CLASSNAME & area\_km2\\
\midrule
\cellcolor{gray!6}{AC} & \cellcolor{gray!6}{2017} & \cellcolor{gray!6}{DESMATAMENTO\_CR} & \cellcolor{gray!6}{158.7}\\
\cmidrule{1-4}
AC & 2017 & CICATRIZ\_DE\_QUEIMADA & 150.5\\
\cmidrule{1-4}
\cellcolor{gray!6}{AC} & \cellcolor{gray!6}{2017} & \cellcolor{gray!6}{DEGRADACAO} & \cellcolor{gray!6}{75.8}\\
\cmidrule{1-4}
AC & 2017 & CS\_DESORDENADO & 45.2\\
\cmidrule{1-4}
\cellcolor{gray!6}{AC} & \cellcolor{gray!6}{2017} & \cellcolor{gray!6}{DESMATAMENTO\_VEG} & \cellcolor{gray!6}{0.2}\\
\cmidrule{1-4}
AC & 2018 & DESMATAMENTO\_CR & 135.1\\
\cmidrule{1-4}
\cellcolor{gray!6}{AC} & \cellcolor{gray!6}{2018} & \cellcolor{gray!6}{DEGRADACAO} & \cellcolor{gray!6}{50.5}\\
\cmidrule{1-4}
AC & 2018 & DESMATAMENTO\_VEG & 31.2\\
\cmidrule{1-4}
\cellcolor{gray!6}{AC} & \cellcolor{gray!6}{2018} & \cellcolor{gray!6}{CS\_DESORDENADO} & \cellcolor{gray!6}{7.7}\\
\cmidrule{1-4}
AC & 2018 & CICATRIZ\_DE\_QUEIMADA & 4.1\\
\cmidrule{1-4}
\cellcolor{gray!6}{AC} & \cellcolor{gray!6}{2019} & \cellcolor{gray!6}{DESMATAMENTO\_CR} & \cellcolor{gray!6}{245.7}\\
\cmidrule{1-4}
AC & 2019 & DEGRADACAO & 66.0\\
\cmidrule{1-4}
\cellcolor{gray!6}{AC} & \cellcolor{gray!6}{2019} & \cellcolor{gray!6}{DESMATAMENTO\_VEG} & \cellcolor{gray!6}{16.0}\\
\cmidrule{1-4}
AC & 2019 & CS\_DESORDENADO & 14.5\\
\cmidrule{1-4}
\cellcolor{gray!6}{AC} & \cellcolor{gray!6}{2019} & \cellcolor{gray!6}{CICATRIZ\_DE\_QUEIMADA} & \cellcolor{gray!6}{3.0}\\
\cmidrule{1-4}
AC & 2020 & DESMATAMENTO\_CR & 450.8\\
\cmidrule{1-4}
\cellcolor{gray!6}{AC} & \cellcolor{gray!6}{2020} & \cellcolor{gray!6}{CS\_DESORDENADO} & \cellcolor{gray!6}{123.9}\\
\cmidrule{1-4}
AC & 2020 & DEGRADACAO & 33.5\\
\cmidrule{1-4}
\cellcolor{gray!6}{AC} & \cellcolor{gray!6}{2020} & \cellcolor{gray!6}{DESMATAMENTO\_VEG} & \cellcolor{gray!6}{23.0}\\
\cmidrule{1-4}
AC & 2020 & CICATRIZ\_DE\_QUEIMADA & 5.5\\
\cmidrule{1-4}
\cellcolor{gray!6}{AC} & \cellcolor{gray!6}{2021} & \cellcolor{gray!6}{DESMATAMENTO\_CR} & \cellcolor{gray!6}{480.7}\\
\cmidrule{1-4}
AC & 2021 & CS\_DESORDENADO & 183.2\\
\cmidrule{1-4}
\cellcolor{gray!6}{AC} & \cellcolor{gray!6}{2021} & \cellcolor{gray!6}{DEGRADACAO} & \cellcolor{gray!6}{27.0}\\
\cmidrule{1-4}
AC & 2021 & DESMATAMENTO\_VEG & 8.7\\
\cmidrule{1-4}
\cellcolor{gray!6}{AC} & \cellcolor{gray!6}{2021} & \cellcolor{gray!6}{CICATRIZ\_DE\_QUEIMADA} & \cellcolor{gray!6}{3.1}\\
\cmidrule{1-4}
AM & 2017 & CICATRIZ\_DE\_QUEIMADA & 1046.6\\
\cmidrule{1-4}
\cellcolor{gray!6}{AM} & \cellcolor{gray!6}{2017} & \cellcolor{gray!6}{DESMATAMENTO\_CR} & \cellcolor{gray!6}{755.9}\\
\cmidrule{1-4}
AM & 2017 & DEGRADACAO & 235.3\\
\cmidrule{1-4}
\cellcolor{gray!6}{AM} & \cellcolor{gray!6}{2017} & \cellcolor{gray!6}{CS\_DESORDENADO} & \cellcolor{gray!6}{149.1}\\
\cmidrule{1-4}
AM & 2017 & CS\_GEOMETRICO & 10.8\\
\cmidrule{1-4}
\cellcolor{gray!6}{AM} & \cellcolor{gray!6}{2017} & \cellcolor{gray!6}{DESMATAMENTO\_VEG} & \cellcolor{gray!6}{9.8}\\
\cmidrule{1-4}
AM & 2017 & CORTE\_SELETIVO & 2.3\\
\cmidrule{1-4}
\cellcolor{gray!6}{AM} & \cellcolor{gray!6}{2017} & \cellcolor{gray!6}{MINERACAO} & \cellcolor{gray!6}{0.7}\\
\cmidrule{1-4}
AM & 2018 & DESMATAMENTO\_CR & 628.8\\
\cmidrule{1-4}
\cellcolor{gray!6}{AM} & \cellcolor{gray!6}{2018} & \cellcolor{gray!6}{DEGRADACAO} & \cellcolor{gray!6}{178.2}\\
\cmidrule{1-4}
AM & 2018 & CS\_DESORDENADO & 63.5\\
\cmidrule{1-4}
\cellcolor{gray!6}{AM} & \cellcolor{gray!6}{2018} & \cellcolor{gray!6}{CICATRIZ\_DE\_QUEIMADA} & \cellcolor{gray!6}{59.7}\\
\cmidrule{1-4}
AM & 2018 & DESMATAMENTO\_VEG & 36.5\\
\cmidrule{1-4}
\cellcolor{gray!6}{AM} & \cellcolor{gray!6}{2018} & \cellcolor{gray!6}{MINERACAO} & \cellcolor{gray!6}{2.5}\\
\cmidrule{1-4}
AM & 2018 & CS\_GEOMETRICO & 1.6\\
\cmidrule{1-4}
\cellcolor{gray!6}{AM} & \cellcolor{gray!6}{2019} & \cellcolor{gray!6}{DESMATAMENTO\_CR} & \cellcolor{gray!6}{1154.3}\\
\cmidrule{1-4}
AM & 2019 & DEGRADACAO & 112.5\\
\cmidrule{1-4}
\cellcolor{gray!6}{AM} & \cellcolor{gray!6}{2019} & \cellcolor{gray!6}{DESMATAMENTO\_VEG} & \cellcolor{gray!6}{66.9}\\
\cmidrule{1-4}
AM & 2019 & CICATRIZ\_DE\_QUEIMADA & 24.7\\
\cmidrule{1-4}
\cellcolor{gray!6}{AM} & \cellcolor{gray!6}{2019} & \cellcolor{gray!6}{CS\_DESORDENADO} & \cellcolor{gray!6}{16.9}\\
\cmidrule{1-4}
AM & 2019 & MINERACAO & 5.5\\
\cmidrule{1-4}
\cellcolor{gray!6}{AM} & \cellcolor{gray!6}{2020} & \cellcolor{gray!6}{DESMATAMENTO\_CR} & \cellcolor{gray!6}{1237.1}\\
\cmidrule{1-4}
AM & 2020 & CS\_DESORDENADO & 320.9\\
\cmidrule{1-4}
\cellcolor{gray!6}{AM} & \cellcolor{gray!6}{2020} & \cellcolor{gray!6}{DEGRADACAO} & \cellcolor{gray!6}{59.5}\\
\cmidrule{1-4}
AM & 2020 & CICATRIZ\_DE\_QUEIMADA & 54.6\\
\cmidrule{1-4}
\cellcolor{gray!6}{AM} & \cellcolor{gray!6}{2020} & \cellcolor{gray!6}{DESMATAMENTO\_VEG} & \cellcolor{gray!6}{30.8}\\
\cmidrule{1-4}
AM & 2020 & CS\_GEOMETRICO & 7.1\\
\cmidrule{1-4}
\cellcolor{gray!6}{AM} & \cellcolor{gray!6}{2020} & \cellcolor{gray!6}{MINERACAO} & \cellcolor{gray!6}{4.4}\\
\cmidrule{1-4}
AM & 2021 & DESMATAMENTO\_CR & 1701.3\\
\cmidrule{1-4}
\cellcolor{gray!6}{AM} & \cellcolor{gray!6}{2021} & \cellcolor{gray!6}{CS\_DESORDENADO} & \cellcolor{gray!6}{201.2}\\
\cmidrule{1-4}
AM & 2021 & DEGRADACAO & 127.0\\
\cmidrule{1-4}
\cellcolor{gray!6}{AM} & \cellcolor{gray!6}{2021} & \cellcolor{gray!6}{CICATRIZ\_DE\_QUEIMADA} & \cellcolor{gray!6}{47.2}\\
\cmidrule{1-4}
AM & 2021 & CS\_GEOMETRICO & 23.4\\
\cmidrule{1-4}
\cellcolor{gray!6}{AM} & \cellcolor{gray!6}{2021} & \cellcolor{gray!6}{DESMATAMENTO\_VEG} & \cellcolor{gray!6}{12.9}\\
\cmidrule{1-4}
AM & 2021 & MINERACAO & 7.6\\
\cmidrule{1-4}
\cellcolor{gray!6}{AP} & \cellcolor{gray!6}{2017} & \cellcolor{gray!6}{CICATRIZ\_DE\_QUEIMADA} & \cellcolor{gray!6}{73.6}\\
\cmidrule{1-4}
AP & 2017 & DEGRADACAO & 33.2\\
\cmidrule{1-4}
\cellcolor{gray!6}{AP} & \cellcolor{gray!6}{2017} & \cellcolor{gray!6}{DESMATAMENTO\_CR} & \cellcolor{gray!6}{6.5}\\
\cmidrule{1-4}
AP & 2017 & DESMATAMENTO\_VEG & 0.2\\
\cmidrule{1-4}
\cellcolor{gray!6}{AP} & \cellcolor{gray!6}{2018} & \cellcolor{gray!6}{DESMATAMENTO\_CR} & \cellcolor{gray!6}{9.9}\\
\cmidrule{1-4}
AP & 2018 & DEGRADACAO & 3.5\\
\cmidrule{1-4}
\cellcolor{gray!6}{AP} & \cellcolor{gray!6}{2018} & \cellcolor{gray!6}{CICATRIZ\_DE\_QUEIMADA} & \cellcolor{gray!6}{2.6}\\
\cmidrule{1-4}
AP & 2018 & DESMATAMENTO\_VEG & 0.7\\
\cmidrule{1-4}
\cellcolor{gray!6}{AP} & \cellcolor{gray!6}{2019} & \cellcolor{gray!6}{DEGRADACAO} & \cellcolor{gray!6}{3.6}\\
\cmidrule{1-4}
AP & 2019 & DESMATAMENTO\_VEG & 3.5\\
\cmidrule{1-4}
\cellcolor{gray!6}{AP} & \cellcolor{gray!6}{2019} & \cellcolor{gray!6}{DESMATAMENTO\_CR} & \cellcolor{gray!6}{1.2}\\
\cmidrule{1-4}
AP & 2019 & MINERACAO & 0.2\\
\cmidrule{1-4}
\cellcolor{gray!6}{AP} & \cellcolor{gray!6}{2020} & \cellcolor{gray!6}{DESMATAMENTO\_CR} & \cellcolor{gray!6}{10.4}\\
\cmidrule{1-4}
AP & 2020 & DESMATAMENTO\_VEG & 2.0\\
\cmidrule{1-4}
\cellcolor{gray!6}{AP} & \cellcolor{gray!6}{2020} & \cellcolor{gray!6}{DEGRADACAO} & \cellcolor{gray!6}{0.8}\\
\cmidrule{1-4}
AP & 2020 & MINERACAO & 0.3\\
\cmidrule{1-4}
\cellcolor{gray!6}{AP} & \cellcolor{gray!6}{2021} & \cellcolor{gray!6}{DESMATAMENTO\_CR} & \cellcolor{gray!6}{1.9}\\
\cmidrule{1-4}
AP & 2021 & CICATRIZ\_DE\_QUEIMADA & 0.4\\
\cmidrule{1-4}
\cellcolor{gray!6}{AP} & \cellcolor{gray!6}{2021} & \cellcolor{gray!6}{DESMATAMENTO\_VEG} & \cellcolor{gray!6}{0.4}\\
\cmidrule{1-4}
AP & 2021 & MINERACAO & 0.3\\
\cmidrule{1-4}
\cellcolor{gray!6}{AP} & \cellcolor{gray!6}{2021} & \cellcolor{gray!6}{DEGRADACAO} & \cellcolor{gray!6}{0.3}\\
\cmidrule{1-4}
MA & 2017 & CICATRIZ\_DE\_QUEIMADA & 2259.6\\
\cmidrule{1-4}
\cellcolor{gray!6}{MA} & \cellcolor{gray!6}{2017} & \cellcolor{gray!6}{DESMATAMENTO\_CR} & \cellcolor{gray!6}{69.2}\\
\cmidrule{1-4}
MA & 2017 & DEGRADACAO & 35.9\\
\cmidrule{1-4}
\cellcolor{gray!6}{MA} & \cellcolor{gray!6}{2017} & \cellcolor{gray!6}{DESMATAMENTO\_VEG} & \cellcolor{gray!6}{4.8}\\
\cmidrule{1-4}
MA & 2018 & CICATRIZ\_DE\_QUEIMADA & 323.7\\
\cmidrule{1-4}
\cellcolor{gray!6}{MA} & \cellcolor{gray!6}{2018} & \cellcolor{gray!6}{DEGRADACAO} & \cellcolor{gray!6}{90.2}\\
\cmidrule{1-4}
MA & 2018 & DESMATAMENTO\_CR & 49.4\\
\cmidrule{1-4}
\cellcolor{gray!6}{MA} & \cellcolor{gray!6}{2018} & \cellcolor{gray!6}{DESMATAMENTO\_VEG} & \cellcolor{gray!6}{1.9}\\
\cmidrule{1-4}
MA & 2018 & MINERACAO & 0.1\\
\cmidrule{1-4}
\cellcolor{gray!6}{MA} & \cellcolor{gray!6}{2019} & \cellcolor{gray!6}{DESMATAMENTO\_CR} & \cellcolor{gray!6}{57.9}\\
\cmidrule{1-4}
MA & 2019 & DEGRADACAO & 11.1\\
\cmidrule{1-4}
\cellcolor{gray!6}{MA} & \cellcolor{gray!6}{2019} & \cellcolor{gray!6}{CICATRIZ\_DE\_QUEIMADA} & \cellcolor{gray!6}{10.9}\\
\cmidrule{1-4}
MA & 2019 & DESMATAMENTO\_VEG & 1.6\\
\cmidrule{1-4}
\cellcolor{gray!6}{MA} & \cellcolor{gray!6}{2020} & \cellcolor{gray!6}{DESMATAMENTO\_CR} & \cellcolor{gray!6}{129.3}\\
\cmidrule{1-4}
MA & 2020 & CICATRIZ\_DE\_QUEIMADA & 113.1\\
\cmidrule{1-4}
\cellcolor{gray!6}{MA} & \cellcolor{gray!6}{2020} & \cellcolor{gray!6}{DEGRADACAO} & \cellcolor{gray!6}{7.8}\\
\cmidrule{1-4}
MA & 2020 & CS\_DESORDENADO & 5.4\\
\cmidrule{1-4}
\cellcolor{gray!6}{MA} & \cellcolor{gray!6}{2020} & \cellcolor{gray!6}{DESMATAMENTO\_VEG} & \cellcolor{gray!6}{1.8}\\
\cmidrule{1-4}
MA & 2020 & MINERACAO & 0.5\\
\cmidrule{1-4}
\cellcolor{gray!6}{MA} & \cellcolor{gray!6}{2021} & \cellcolor{gray!6}{DESMATAMENTO\_CR} & \cellcolor{gray!6}{115.9}\\
\cmidrule{1-4}
MA & 2021 & CICATRIZ\_DE\_QUEIMADA & 100.2\\
\cmidrule{1-4}
\cellcolor{gray!6}{MA} & \cellcolor{gray!6}{2021} & \cellcolor{gray!6}{DEGRADACAO} & \cellcolor{gray!6}{5.2}\\
\cmidrule{1-4}
MA & 2021 & DESMATAMENTO\_VEG & 1.5\\
\cmidrule{1-4}
\cellcolor{gray!6}{MA} & \cellcolor{gray!6}{2021} & \cellcolor{gray!6}{MINERACAO} & \cellcolor{gray!6}{0.7}\\
\cmidrule{1-4}
MA & 2021 & CS\_DESORDENADO & 0.5\\
\cmidrule{1-4}
\cellcolor{gray!6}{MT} & \cellcolor{gray!6}{2017} & \cellcolor{gray!6}{CICATRIZ\_DE\_QUEIMADA} & \cellcolor{gray!6}{3850.8}\\
\cmidrule{1-4}
MT & 2017 & DEGRADACAO & 1198.5\\
\cmidrule{1-4}
\cellcolor{gray!6}{MT} & \cellcolor{gray!6}{2017} & \cellcolor{gray!6}{DESMATAMENTO\_CR} & \cellcolor{gray!6}{1160.4}\\
\cmidrule{1-4}
MT & 2017 & CS\_GEOMETRICO & 394.0\\
\cmidrule{1-4}
\cellcolor{gray!6}{MT} & \cellcolor{gray!6}{2017} & \cellcolor{gray!6}{CS\_DESORDENADO} & \cellcolor{gray!6}{344.3}\\
\cmidrule{1-4}
MT & 2017 & DESMATAMENTO\_VEG & 9.2\\
\cmidrule{1-4}
\cellcolor{gray!6}{MT} & \cellcolor{gray!6}{2017} & \cellcolor{gray!6}{MINERACAO} & \cellcolor{gray!6}{5.6}\\
\cmidrule{1-4}
MT & 2018 & CICATRIZ\_DE\_QUEIMADA & 7314.3\\
\cmidrule{1-4}
\cellcolor{gray!6}{MT} & \cellcolor{gray!6}{2018} & \cellcolor{gray!6}{DEGRADACAO} & \cellcolor{gray!6}{1442.1}\\
\cmidrule{1-4}
MT & 2018 & DESMATAMENTO\_CR & 1087.0\\
\cmidrule{1-4}
\cellcolor{gray!6}{MT} & \cellcolor{gray!6}{2018} & \cellcolor{gray!6}{CS\_GEOMETRICO} & \cellcolor{gray!6}{622.7}\\
\cmidrule{1-4}
MT & 2018 & CS\_DESORDENADO & 401.3\\
\cmidrule{1-4}
\cellcolor{gray!6}{MT} & \cellcolor{gray!6}{2018} & \cellcolor{gray!6}{DESMATAMENTO\_VEG} & \cellcolor{gray!6}{51.1}\\
\cmidrule{1-4}
MT & 2018 & MINERACAO & 6.3\\
\cmidrule{1-4}
\cellcolor{gray!6}{MT} & \cellcolor{gray!6}{2019} & \cellcolor{gray!6}{CICATRIZ\_DE\_QUEIMADA} & \cellcolor{gray!6}{2107.7}\\
\cmidrule{1-4}
MT & 2019 & DESMATAMENTO\_CR & 1337.2\\
\cmidrule{1-4}
\cellcolor{gray!6}{MT} & \cellcolor{gray!6}{2019} & \cellcolor{gray!6}{CS\_GEOMETRICO} & \cellcolor{gray!6}{914.3}\\
\cmidrule{1-4}
MT & 2019 & DEGRADACAO & 668.4\\
\cmidrule{1-4}
\cellcolor{gray!6}{MT} & \cellcolor{gray!6}{2019} & \cellcolor{gray!6}{CS\_DESORDENADO} & \cellcolor{gray!6}{198.2}\\
\cmidrule{1-4}
MT & 2019 & DESMATAMENTO\_VEG & 53.4\\
\cmidrule{1-4}
\cellcolor{gray!6}{MT} & \cellcolor{gray!6}{2019} & \cellcolor{gray!6}{MINERACAO} & \cellcolor{gray!6}{7.4}\\
\cmidrule{1-4}
MT & 2020 & CICATRIZ\_DE\_QUEIMADA & 3565.9\\
\cmidrule{1-4}
\cellcolor{gray!6}{MT} & \cellcolor{gray!6}{2020} & \cellcolor{gray!6}{DESMATAMENTO\_CR} & \cellcolor{gray!6}{1848.8}\\
\cmidrule{1-4}
MT & 2020 & CS\_GEOMETRICO & 1108.4\\
\cmidrule{1-4}
\cellcolor{gray!6}{MT} & \cellcolor{gray!6}{2020} & \cellcolor{gray!6}{DEGRADACAO} & \cellcolor{gray!6}{867.2}\\
\cmidrule{1-4}
MT & 2020 & CS\_DESORDENADO & 586.0\\
\cmidrule{1-4}
\cellcolor{gray!6}{MT} & \cellcolor{gray!6}{2020} & \cellcolor{gray!6}{DESMATAMENTO\_VEG} & \cellcolor{gray!6}{20.2}\\
\cmidrule{1-4}
MT & 2020 & MINERACAO & 6.0\\
\cmidrule{1-4}
\cellcolor{gray!6}{MT} & \cellcolor{gray!6}{2021} & \cellcolor{gray!6}{CICATRIZ\_DE\_QUEIMADA} & \cellcolor{gray!6}{8382.0}\\
\cmidrule{1-4}
MT & 2021 & DESMATAMENTO\_CR & 1464.3\\
\cmidrule{1-4}
\cellcolor{gray!6}{MT} & \cellcolor{gray!6}{2021} & \cellcolor{gray!6}{CS\_DESORDENADO} & \cellcolor{gray!6}{1138.1}\\
\cmidrule{1-4}
MT & 2021 & CS\_GEOMETRICO & 1074.1\\
\cmidrule{1-4}
\cellcolor{gray!6}{MT} & \cellcolor{gray!6}{2021} & \cellcolor{gray!6}{DEGRADACAO} & \cellcolor{gray!6}{708.7}\\
\cmidrule{1-4}
MT & 2021 & DESMATAMENTO\_VEG & 13.0\\
\cmidrule{1-4}
\cellcolor{gray!6}{MT} & \cellcolor{gray!6}{2021} & \cellcolor{gray!6}{MINERACAO} & \cellcolor{gray!6}{8.9}\\
\cmidrule{1-4}
PA & 2017 & CICATRIZ\_DE\_QUEIMADA & 4456.4\\
\cmidrule{1-4}
\cellcolor{gray!6}{PA} & \cellcolor{gray!6}{2017} & \cellcolor{gray!6}{DEGRADACAO} & \cellcolor{gray!6}{1793.4}\\
\cmidrule{1-4}
PA & 2017 & DESMATAMENTO\_CR & 1260.4\\
\cmidrule{1-4}
\cellcolor{gray!6}{PA} & \cellcolor{gray!6}{2017} & \cellcolor{gray!6}{CS\_DESORDENADO} & \cellcolor{gray!6}{245.1}\\
\cmidrule{1-4}
PA & 2017 & CS\_GEOMETRICO & 161.7\\
\cmidrule{1-4}
\cellcolor{gray!6}{PA} & \cellcolor{gray!6}{2017} & \cellcolor{gray!6}{DESMATAMENTO\_VEG} & \cellcolor{gray!6}{115.8}\\
\cmidrule{1-4}
PA & 2017 & MINERACAO & 48.4\\
\cmidrule{1-4}
\cellcolor{gray!6}{PA} & \cellcolor{gray!6}{2018} & \cellcolor{gray!6}{CICATRIZ\_DE\_QUEIMADA} & \cellcolor{gray!6}{8820.7}\\
\cmidrule{1-4}
PA & 2018 & DEGRADACAO & 1536.8\\
\cmidrule{1-4}
\cellcolor{gray!6}{PA} & \cellcolor{gray!6}{2018} & \cellcolor{gray!6}{DESMATAMENTO\_CR} & \cellcolor{gray!6}{1361.0}\\
\cmidrule{1-4}
PA & 2018 & CS\_DESORDENADO & 218.2\\
\cmidrule{1-4}
\cellcolor{gray!6}{PA} & \cellcolor{gray!6}{2018} & \cellcolor{gray!6}{CS\_GEOMETRICO} & \cellcolor{gray!6}{201.6}\\
\cmidrule{1-4}
PA & 2018 & DESMATAMENTO\_VEG & 118.2\\
\cmidrule{1-4}
\cellcolor{gray!6}{PA} & \cellcolor{gray!6}{2018} & \cellcolor{gray!6}{MINERACAO} & \cellcolor{gray!6}{67.1}\\
\cmidrule{1-4}
PA & 2019 & DESMATAMENTO\_CR & 2032.0\\
\cmidrule{1-4}
\cellcolor{gray!6}{PA} & \cellcolor{gray!6}{2019} & \cellcolor{gray!6}{DEGRADACAO} & \cellcolor{gray!6}{562.9}\\
\cmidrule{1-4}
PA & 2019 & DESMATAMENTO\_VEG & 287.4\\
\cmidrule{1-4}
\cellcolor{gray!6}{PA} & \cellcolor{gray!6}{2019} & \cellcolor{gray!6}{CS\_GEOMETRICO} & \cellcolor{gray!6}{192.9}\\
\cmidrule{1-4}
PA & 2019 & CICATRIZ\_DE\_QUEIMADA & 148.5\\
\cmidrule{1-4}
\cellcolor{gray!6}{PA} & \cellcolor{gray!6}{2019} & \cellcolor{gray!6}{CS\_DESORDENADO} & \cellcolor{gray!6}{96.1}\\
\cmidrule{1-4}
PA & 2019 & MINERACAO & 88.0\\
\cmidrule{1-4}
\cellcolor{gray!6}{PA} & \cellcolor{gray!6}{2020} & \cellcolor{gray!6}{DESMATAMENTO\_CR} & \cellcolor{gray!6}{3674.5}\\
\cmidrule{1-4}
PA & 2020 & CICATRIZ\_DE\_QUEIMADA & 2067.9\\
\cmidrule{1-4}
\cellcolor{gray!6}{PA} & \cellcolor{gray!6}{2020} & \cellcolor{gray!6}{CS\_DESORDENADO} & \cellcolor{gray!6}{840.0}\\
\cmidrule{1-4}
PA & 2020 & DEGRADACAO & 618.1\\
\cmidrule{1-4}
\cellcolor{gray!6}{PA} & \cellcolor{gray!6}{2020} & \cellcolor{gray!6}{CS\_GEOMETRICO} & \cellcolor{gray!6}{272.5}\\
\cmidrule{1-4}
PA & 2020 & DESMATAMENTO\_VEG & 160.4\\
\cmidrule{1-4}
\cellcolor{gray!6}{PA} & \cellcolor{gray!6}{2020} & \cellcolor{gray!6}{MINERACAO} & \cellcolor{gray!6}{80.9}\\
\cmidrule{1-4}
PA & 2021 & CICATRIZ\_DE\_QUEIMADA & 3903.2\\
\cmidrule{1-4}
\cellcolor{gray!6}{PA} & \cellcolor{gray!6}{2021} & \cellcolor{gray!6}{DESMATAMENTO\_CR} & \cellcolor{gray!6}{3271.0}\\
\cmidrule{1-4}
PA & 2021 & CS\_DESORDENADO & 779.8\\
\cmidrule{1-4}
\cellcolor{gray!6}{PA} & \cellcolor{gray!6}{2021} & \cellcolor{gray!6}{DEGRADACAO} & \cellcolor{gray!6}{741.5}\\
\cmidrule{1-4}
PA & 2021 & CS\_GEOMETRICO & 402.3\\
\cmidrule{1-4}
\cellcolor{gray!6}{PA} & \cellcolor{gray!6}{2021} & \cellcolor{gray!6}{MINERACAO} & \cellcolor{gray!6}{105.7}\\
\cmidrule{1-4}
PA & 2021 & DESMATAMENTO\_VEG & 69.2\\
\cmidrule{1-4}
\cellcolor{gray!6}{RO} & \cellcolor{gray!6}{2017} & \cellcolor{gray!6}{DESMATAMENTO\_CR} & \cellcolor{gray!6}{833.3}\\
\cmidrule{1-4}
RO & 2017 & DEGRADACAO & 217.8\\
\cmidrule{1-4}
\cellcolor{gray!6}{RO} & \cellcolor{gray!6}{2017} & \cellcolor{gray!6}{CICATRIZ\_DE\_QUEIMADA} & \cellcolor{gray!6}{179.0}\\
\cmidrule{1-4}
RO & 2017 & CS\_DESORDENADO & 111.8\\
\cmidrule{1-4}
\cellcolor{gray!6}{RO} & \cellcolor{gray!6}{2017} & \cellcolor{gray!6}{CS\_GEOMETRICO} & \cellcolor{gray!6}{41.1}\\
\cmidrule{1-4}
RO & 2017 & DESMATAMENTO\_VEG & 17.2\\
\cmidrule{1-4}
\cellcolor{gray!6}{RO} & \cellcolor{gray!6}{2017} & \cellcolor{gray!6}{MINERACAO} & \cellcolor{gray!6}{0.2}\\
\cmidrule{1-4}
RO & 2018 & DESMATAMENTO\_CR & 685.6\\
\cmidrule{1-4}
\cellcolor{gray!6}{RO} & \cellcolor{gray!6}{2018} & \cellcolor{gray!6}{DEGRADACAO} & \cellcolor{gray!6}{230.4}\\
\cmidrule{1-4}
RO & 2018 & CS\_DESORDENADO & 148.8\\
\cmidrule{1-4}
\cellcolor{gray!6}{RO} & \cellcolor{gray!6}{2018} & \cellcolor{gray!6}{CICATRIZ\_DE\_QUEIMADA} & \cellcolor{gray!6}{92.9}\\
\cmidrule{1-4}
RO & 2018 & DESMATAMENTO\_VEG & 50.1\\
\cmidrule{1-4}
\cellcolor{gray!6}{RO} & \cellcolor{gray!6}{2018} & \cellcolor{gray!6}{CS\_GEOMETRICO} & \cellcolor{gray!6}{34.2}\\
\cmidrule{1-4}
RO & 2018 & MINERACAO & 1.1\\
\cmidrule{1-4}
\cellcolor{gray!6}{RO} & \cellcolor{gray!6}{2019} & \cellcolor{gray!6}{DESMATAMENTO\_CR} & \cellcolor{gray!6}{1024.4}\\
\cmidrule{1-4}
RO & 2019 & DEGRADACAO & 155.6\\
\cmidrule{1-4}
\cellcolor{gray!6}{RO} & \cellcolor{gray!6}{2019} & \cellcolor{gray!6}{CICATRIZ\_DE\_QUEIMADA} & \cellcolor{gray!6}{78.7}\\
\cmidrule{1-4}
RO & 2019 & CS\_GEOMETRICO & 41.8\\
\cmidrule{1-4}
\cellcolor{gray!6}{RO} & \cellcolor{gray!6}{2019} & \cellcolor{gray!6}{DESMATAMENTO\_VEG} & \cellcolor{gray!6}{38.1}\\
\cmidrule{1-4}
RO & 2019 & CS\_DESORDENADO & 3.7\\
\cmidrule{1-4}
\cellcolor{gray!6}{RO} & \cellcolor{gray!6}{2019} & \cellcolor{gray!6}{MINERACAO} & \cellcolor{gray!6}{1.0}\\
\cmidrule{1-4}
RO & 2020 & DESMATAMENTO\_CR & 1258.2\\
\cmidrule{1-4}
\cellcolor{gray!6}{RO} & \cellcolor{gray!6}{2020} & \cellcolor{gray!6}{CICATRIZ\_DE\_QUEIMADA} & \cellcolor{gray!6}{218.1}\\
\cmidrule{1-4}
RO & 2020 & DEGRADACAO & 119.4\\
\cmidrule{1-4}
\cellcolor{gray!6}{RO} & \cellcolor{gray!6}{2020} & \cellcolor{gray!6}{CS\_DESORDENADO} & \cellcolor{gray!6}{54.1}\\
\cmidrule{1-4}
RO & 2020 & CS\_GEOMETRICO & 28.6\\
\cmidrule{1-4}
\cellcolor{gray!6}{RO} & \cellcolor{gray!6}{2020} & \cellcolor{gray!6}{DESMATAMENTO\_VEG} & \cellcolor{gray!6}{12.4}\\
\cmidrule{1-4}
RO & 2020 & MINERACAO & 1.2\\
\cmidrule{1-4}
\cellcolor{gray!6}{RO} & \cellcolor{gray!6}{2021} & \cellcolor{gray!6}{DESMATAMENTO\_CR} & \cellcolor{gray!6}{1321.6}\\
\cmidrule{1-4}
RO & 2021 & CICATRIZ\_DE\_QUEIMADA & 332.0\\
\cmidrule{1-4}
\cellcolor{gray!6}{RO} & \cellcolor{gray!6}{2021} & \cellcolor{gray!6}{DEGRADACAO} & \cellcolor{gray!6}{160.1}\\
\cmidrule{1-4}
RO & 2021 & CS\_DESORDENADO & 145.1\\
\cmidrule{1-4}
\cellcolor{gray!6}{RO} & \cellcolor{gray!6}{2021} & \cellcolor{gray!6}{CS\_GEOMETRICO} & \cellcolor{gray!6}{13.7}\\
\cmidrule{1-4}
RO & 2021 & DESMATAMENTO\_VEG & 7.6\\
\cmidrule{1-4}
\cellcolor{gray!6}{RO} & \cellcolor{gray!6}{2021} & \cellcolor{gray!6}{MINERACAO} & \cellcolor{gray!6}{1.7}\\
\cmidrule{1-4}
RR & 2017 & CICATRIZ\_DE\_QUEIMADA & 3096.7\\
\cmidrule{1-4}
\cellcolor{gray!6}{RR} & \cellcolor{gray!6}{2017} & \cellcolor{gray!6}{DEGRADACAO} & \cellcolor{gray!6}{255.6}\\
\cmidrule{1-4}
RR & 2017 & DESMATAMENTO\_CR & 51.0\\
\cmidrule{1-4}
\cellcolor{gray!6}{RR} & \cellcolor{gray!6}{2017} & \cellcolor{gray!6}{DESMATAMENTO\_VEG} & \cellcolor{gray!6}{2.1}\\
\cmidrule{1-4}
RR & 2018 & CICATRIZ\_DE\_QUEIMADA & 188.7\\
\cmidrule{1-4}
\cellcolor{gray!6}{RR} & \cellcolor{gray!6}{2018} & \cellcolor{gray!6}{DEGRADACAO} & \cellcolor{gray!6}{168.8}\\
\cmidrule{1-4}
RR & 2018 & DESMATAMENTO\_CR & 149.3\\
\cmidrule{1-4}
\cellcolor{gray!6}{RR} & \cellcolor{gray!6}{2018} & \cellcolor{gray!6}{CS\_DESORDENADO} & \cellcolor{gray!6}{26.3}\\
\cmidrule{1-4}
RR & 2018 & DESMATAMENTO\_VEG & 7.2\\
\cmidrule{1-4}
\cellcolor{gray!6}{RR} & \cellcolor{gray!6}{2018} & \cellcolor{gray!6}{CS\_GEOMETRICO} & \cellcolor{gray!6}{2.9}\\
\cmidrule{1-4}
RR & 2019 & CICATRIZ\_DE\_QUEIMADA & 1242.4\\
\cmidrule{1-4}
\cellcolor{gray!6}{RR} & \cellcolor{gray!6}{2019} & \cellcolor{gray!6}{DESMATAMENTO\_CR} & \cellcolor{gray!6}{221.7}\\
\cmidrule{1-4}
RR & 2019 & DEGRADACAO & 69.5\\
\cmidrule{1-4}
\cellcolor{gray!6}{RR} & \cellcolor{gray!6}{2019} & \cellcolor{gray!6}{CS\_DESORDENADO} & \cellcolor{gray!6}{16.6}\\
\cmidrule{1-4}
RR & 2019 & CS\_GEOMETRICO & 15.9\\
\cmidrule{1-4}
\cellcolor{gray!6}{RR} & \cellcolor{gray!6}{2019} & \cellcolor{gray!6}{DESMATAMENTO\_VEG} & \cellcolor{gray!6}{10.4}\\
\cmidrule{1-4}
RR & 2019 & MINERACAO & 1.1\\
\cmidrule{1-4}
\cellcolor{gray!6}{RR} & \cellcolor{gray!6}{2020} & \cellcolor{gray!6}{DESMATAMENTO\_CR} & \cellcolor{gray!6}{218.3}\\
\cmidrule{1-4}
RR & 2020 & CICATRIZ\_DE\_QUEIMADA & 83.2\\
\cmidrule{1-4}
\cellcolor{gray!6}{RR} & \cellcolor{gray!6}{2020} & \cellcolor{gray!6}{DEGRADACAO} & \cellcolor{gray!6}{82.9}\\
\cmidrule{1-4}
RR & 2020 & DESMATAMENTO\_VEG & 21.6\\
\cmidrule{1-4}
\cellcolor{gray!6}{RR} & \cellcolor{gray!6}{2020} & \cellcolor{gray!6}{CS\_DESORDENADO} & \cellcolor{gray!6}{18.5}\\
\cmidrule{1-4}
RR & 2020 & CS\_GEOMETRICO & 10.6\\
\cmidrule{1-4}
\cellcolor{gray!6}{RR} & \cellcolor{gray!6}{2020} & \cellcolor{gray!6}{MINERACAO} & \cellcolor{gray!6}{1.4}\\
\cmidrule{1-4}
RR & 2021 & DESMATAMENTO\_CR & 155.4\\
\cmidrule{1-4}
\cellcolor{gray!6}{RR} & \cellcolor{gray!6}{2021} & \cellcolor{gray!6}{CS\_DESORDENADO} & \cellcolor{gray!6}{33.1}\\
\cmidrule{1-4}
RR & 2021 & DEGRADACAO & 17.0\\
\cmidrule{1-4}
\cellcolor{gray!6}{RR} & \cellcolor{gray!6}{2021} & \cellcolor{gray!6}{DESMATAMENTO\_VEG} & \cellcolor{gray!6}{5.3}\\
\cmidrule{1-4}
RR & 2021 & CICATRIZ\_DE\_QUEIMADA & 4.1\\
\cmidrule{1-4}
\cellcolor{gray!6}{RR} & \cellcolor{gray!6}{2021} & \cellcolor{gray!6}{CS\_GEOMETRICO} & \cellcolor{gray!6}{1.9}\\
\cmidrule{1-4}
RR & 2021 & MINERACAO & 1.1\\
\cmidrule{1-4}
\cellcolor{gray!6}{TO} & \cellcolor{gray!6}{2017} & \cellcolor{gray!6}{CICATRIZ\_DE\_QUEIMADA} & \cellcolor{gray!6}{36.8}\\
\cmidrule{1-4}
TO & 2017 & DESMATAMENTO\_CR & 7.0\\
\cmidrule{1-4}
\cellcolor{gray!6}{TO} & \cellcolor{gray!6}{2017} & \cellcolor{gray!6}{DEGRADACAO} & \cellcolor{gray!6}{1.8}\\
\cmidrule{1-4}
TO & 2017 & DESMATAMENTO\_VEG & 0.4\\
\cmidrule{1-4}
\cellcolor{gray!6}{TO} & \cellcolor{gray!6}{2018} & \cellcolor{gray!6}{CICATRIZ\_DE\_QUEIMADA} & \cellcolor{gray!6}{143.9}\\
\cmidrule{1-4}
TO & 2018 & DESMATAMENTO\_CR & 4.5\\
\cmidrule{1-4}
\cellcolor{gray!6}{TO} & \cellcolor{gray!6}{2018} & \cellcolor{gray!6}{DEGRADACAO} & \cellcolor{gray!6}{1.9}\\
\cmidrule{1-4}
TO & 2019 & DESMATAMENTO\_CR & 3.1\\
\cmidrule{1-4}
\cellcolor{gray!6}{TO} & \cellcolor{gray!6}{2019} & \cellcolor{gray!6}{DEGRADACAO} & \cellcolor{gray!6}{0.5}\\
\cmidrule{1-4}
TO & 2019 & CICATRIZ\_DE\_QUEIMADA & 0.3\\
\cmidrule{1-4}
\cellcolor{gray!6}{TO} & \cellcolor{gray!6}{2020} & \cellcolor{gray!6}{CICATRIZ\_DE\_QUEIMADA} & \cellcolor{gray!6}{9.9}\\
\cmidrule{1-4}
TO & 2020 & DEGRADACAO & 8.2\\
\cmidrule{1-4}
\cellcolor{gray!6}{TO} & \cellcolor{gray!6}{2020} & \cellcolor{gray!6}{DESMATAMENTO\_CR} & \cellcolor{gray!6}{6.1}\\
\cmidrule{1-4}
TO & 2021 & CICATRIZ\_DE\_QUEIMADA & 40.8\\
\cmidrule{1-4}
\cellcolor{gray!6}{TO} & \cellcolor{gray!6}{2021} & \cellcolor{gray!6}{DEGRADACAO} & \cellcolor{gray!6}{6.5}\\
\cmidrule{1-4}
TO & 2021 & DESMATAMENTO\_CR & 6.5\\
\cmidrule{1-4}
\cellcolor{gray!6}{TO} & \cellcolor{gray!6}{2021} & \cellcolor{gray!6}{CS\_DESORDENADO} & \cellcolor{gray!6}{6.1}\\
\cmidrule{1-4}
Total & - & - & 113185.1\\
\bottomrule
\end{longtabu}
\endgroup{}
% \end{frame}

\begin{frame}
    \frametitle{DETER warnings by class}
    \begin{figure}[h]
        \includegraphics[width=0.65\linewidth]
        {./figures/plot_deter_area_by_class.png}
        \label{fig:deter_area_by_class}
        \caption{Burn scars and clear cut are the most common warnings.}
    \end{figure}
\end{frame}

\begin{frame}
    \frametitle{DETER warnings by class and state}
    \begin{figure}[h]
        \includegraphics[width=0.65\linewidth]
        {./figures/plot_deter_area_by_class_state.png}
        \label{fig:deter_area_by_class_state}
        \caption{Burn scars and clear cut are the most common warnings.}
    \end{figure}
\end{frame}


\begin{frame}
    \frametitle{DETER warnings and time}
    \begin{itemize}
        \item The spatial properties of DETER warning are inconsistent along 
            time (shape, size, position, orientation).
    \end{itemize}
\end{frame}

\begin{frame}
    \frametitle{Warnings are inconsistent along time}
    \begin{figure}[h] 
        \includegraphics[width=0.60\linewidth]
        {./images/sample_deter_warnings.png}
        \label{fig:deter_subareas}
        \caption{DETER warnings don't fit along time.}
    \end{figure}
\end{frame}



\subsection{DETER subareas}

\begin{frame}
    \frametitle{DETER subareas}
    \begin{itemize}
        \item The spatial properties of DETER warning are inconsistent along 
            time (shape, size, position, orientation).
        \item DETER subareas maintain their spatial properties along time.
    \end{itemize}
\end{frame}

\begin{frame}
    \frametitle{DETER subareas}
    \begin{figure}[h] 
        \includegraphics[width=0.60\linewidth]
        {./images/sample_deter_subareas.png}
        \caption{From 3 DETER warnings, we get 7 subareas!}
        \label{fig:deter_subareas}
    \end{figure}
\end{frame}





\section{DETER subareas EDA}

\begin{frame}
    \frametitle{DETER subareas}
    \begin{figure}[h] 
        \includegraphics[width=0.65\linewidth]
        {./figures/plot_deter_subarea_by_nwarnings.png}
        \caption{There are subareas with up to 5 recurrent warnings.}
        \label{fig:deter_subareas_nwarnings}
    \end{figure}
\end{frame}

\begin{frame}
    \frametitle{DETER subareas}
    \begin{figure}[h] 
        \includegraphics[width=0.65\linewidth]
        {./figures/plot_deter_subarea_by_warnings_state.png}
        \caption{The warning recurrence changes by brazilian state.}
        \label{fig:deter_subarea_warnings_state}
    \end{figure}
\end{frame}

\begin{frame}
    \frametitle{DETER subareas}
    \begin{figure}[h] 
        \includegraphics[width=0.65\linewidth]
        {./figures/plot_deter_days_first_to_last.png}
        \caption{Number of days between first and last warning.}
        \label{fig:deter_days_first_to_last}
    \end{figure}
\end{frame}

\begin{frame}
    \frametitle{DETER subareas}
    \begin{figure}[h] 
        \includegraphics[width=0.65\linewidth]
        {./figures/plot_deter_subarea_density_by_state_first-type_nwarnings.png}
        \caption{The number of days between warnings behavious in space and 
        time.}
        \label{fig:deter_subarea_density_state_first_type_nwarnings}
    \end{figure}
\end{frame}





\section{Subareas trajectories} 

\begin{frame}
    \frametitle{Subarea trajectories}
    \begin{itemize}
        \item Overlaped subareas organized along time describe trajectories of
            change.
    \end{itemize}
\end{frame}


\subsection{Trajectories (DETER)} 

\begin{frame}
    \frametitle{DETER subareas (2 warnings)}
    \begin{figure}[h] 
        \includegraphics[width=0.65\linewidth]
        {./figures/plot_deter_subarea_trajectory_2.png}
        \caption{Tajectory of subareas with 2 wanings.}
        \label{fig:deter_subarea_trajectory_2}
    \end{figure}
\end{frame}

\begin{frame}[allowframebreaks]
    \frametitle{DETER - Top 5 trajectories (2 warnings)}
    \begingroup\fontsize{7}{9}\selectfont

\begin{longtabu} to \linewidth {>{\raggedright}X>{\raggedright}X>{\raggedleft}X>{\raggedleft}X>{\raggedleft}X>{\raggedleft}X}
\toprule
position\_1 & position\_2 & area\_ha & n\_traj & p\_area & p\_traj\\
\midrule
\cellcolor{gray!6}{Cicatriz de queimada} & \cellcolor{gray!6}{Cicatriz de queimada} & \cellcolor{gray!6}{673806.1} & \cellcolor{gray!6}{15015} & \cellcolor{gray!6}{54.5} & \cellcolor{gray!6}{33.5}\\
\cmidrule{1-6}
Cicatriz de queimada & Desmatamento cr & 89540.6 & 5493 & 7.2 & 12.3\\
\cmidrule{1-6}
\cellcolor{gray!6}{Desmatamento cr} & \cellcolor{gray!6}{Desmatamento cr} & \cellcolor{gray!6}{71670.7} & \cellcolor{gray!6}{7882} & \cellcolor{gray!6}{5.8} & \cellcolor{gray!6}{17.6}\\
\cmidrule{1-6}
Cs geometrico & Cs geometrico & 53594.9 & 623 & 4.3 & 1.4\\
\cmidrule{1-6}
\cellcolor{gray!6}{Degradacao} & \cellcolor{gray!6}{Desmatamento cr} & \cellcolor{gray!6}{52004.1} & \cellcolor{gray!6}{3935} & \cellcolor{gray!6}{4.2} & \cellcolor{gray!6}{8.8}\\
\cmidrule{1-6}
Total & - & 1236329.4 & 44831 & 100.0 & 100.0\\
\bottomrule
\end{longtabu}
\endgroup{}
\end{frame}

\begin{frame}
    \frametitle{DETER subareas (3 warnings)}
    \begin{figure}[h] 
        \includegraphics[width=0.65\linewidth]
        {./figures/plot_deter_subarea_trajectory_3.png}
        \caption{Tajectory of subareas with 3 wanings.}
        \label{fig:deter_subarea_trajectory_3}
    \end{figure}
\end{frame}

\begin{frame}[allowframebreaks]
    \frametitle{DETER - Top 5 trajectories (3 warnings)}
    \begingroup\fontsize{7}{9}\selectfont

\begin{longtabu} to \linewidth {>{\raggedright}X>{\raggedright}X>{\raggedright}X>{\raggedleft}X>{\raggedleft}X}
\toprule
position\_1 & position\_2 & position\_3 & n\_traj & subarea\_ha\\
\midrule
\cellcolor{gray!6}{Cicatriz de queimada} & \cellcolor{gray!6}{Cicatriz de queimada} & \cellcolor{gray!6}{Cicatriz de queimada} & \cellcolor{gray!6}{1789} & \cellcolor{gray!6}{96013.6}\\
\cmidrule{1-5}
Cicatriz de queimada & Cicatriz de queimada & Cs desordenado & 25 & 678.8\\
\cmidrule{1-5}
\cellcolor{gray!6}{Cicatriz de queimada} & \cellcolor{gray!6}{Cicatriz de queimada} & \cellcolor{gray!6}{Degradacao} & \cellcolor{gray!6}{345} & \cellcolor{gray!6}{11700.9}\\
\cmidrule{1-5}
Cicatriz de queimada & Cicatriz de queimada & Desmatamento cr & 423 & 8944.6\\
\cmidrule{1-5}
\cellcolor{gray!6}{Cicatriz de queimada} & \cellcolor{gray!6}{Cicatriz de queimada} & \cellcolor{gray!6}{Desmatamento veg} & \cellcolor{gray!6}{12} & \cellcolor{gray!6}{201.0}\\
\cmidrule{1-5}
Cicatriz de queimada & Cicatriz de queimada & Mineracao & 2 & 7.1\\
\cmidrule{1-5}
\cellcolor{gray!6}{Cicatriz de queimada} & \cellcolor{gray!6}{Cs desordenado} & \cellcolor{gray!6}{Cicatriz de queimada} & \cellcolor{gray!6}{25} & \cellcolor{gray!6}{596.7}\\
\cmidrule{1-5}
Cicatriz de queimada & Cs desordenado & Cs desordenado & 14 & 333.8\\
\cmidrule{1-5}
\cellcolor{gray!6}{Cicatriz de queimada} & \cellcolor{gray!6}{Cs desordenado} & \cellcolor{gray!6}{Cs geometrico} & \cellcolor{gray!6}{1} & \cellcolor{gray!6}{3.5}\\
\cmidrule{1-5}
Cicatriz de queimada & Cs desordenado & Degradacao & 3 & 21.2\\
\cmidrule{1-5}
\cellcolor{gray!6}{Cicatriz de queimada} & \cellcolor{gray!6}{Cs desordenado} & \cellcolor{gray!6}{Desmatamento cr} & \cellcolor{gray!6}{9} & \cellcolor{gray!6}{87.9}\\
\cmidrule{1-5}
Cicatriz de queimada & Cs geometrico & Cicatriz de queimada & 6 & 865.4\\
\cmidrule{1-5}
\cellcolor{gray!6}{Cicatriz de queimada} & \cellcolor{gray!6}{Cs geometrico} & \cellcolor{gray!6}{Cs geometrico} & \cellcolor{gray!6}{11} & \cellcolor{gray!6}{395.9}\\
\cmidrule{1-5}
Cicatriz de queimada & Cs geometrico & Degradacao & 6 & 254.5\\
\cmidrule{1-5}
\cellcolor{gray!6}{Cicatriz de queimada} & \cellcolor{gray!6}{Cs geometrico} & \cellcolor{gray!6}{Desmatamento cr} & \cellcolor{gray!6}{2} & \cellcolor{gray!6}{142.3}\\
\cmidrule{1-5}
Cicatriz de queimada & Degradacao & Cicatriz de queimada & 336 & 11374.3\\
\cmidrule{1-5}
\cellcolor{gray!6}{Cicatriz de queimada} & \cellcolor{gray!6}{Degradacao} & \cellcolor{gray!6}{Cs desordenado} & \cellcolor{gray!6}{10} & \cellcolor{gray!6}{134.9}\\
\cmidrule{1-5}
Cicatriz de queimada & Degradacao & Cs geometrico & 4 & 45.3\\
\cmidrule{1-5}
\cellcolor{gray!6}{Cicatriz de queimada} & \cellcolor{gray!6}{Degradacao} & \cellcolor{gray!6}{Degradacao} & \cellcolor{gray!6}{92} & \cellcolor{gray!6}{2481.7}\\
\cmidrule{1-5}
Cicatriz de queimada & Degradacao & Desmatamento cr & 173 & 2353.2\\
\cmidrule{1-5}
\cellcolor{gray!6}{Cicatriz de queimada} & \cellcolor{gray!6}{Degradacao} & \cellcolor{gray!6}{Desmatamento veg} & \cellcolor{gray!6}{2} & \cellcolor{gray!6}{38.3}\\
\cmidrule{1-5}
Cicatriz de queimada & Desmatamento cr & Cicatriz de queimada & 144 & 2152.1\\
\cmidrule{1-5}
\cellcolor{gray!6}{Cicatriz de queimada} & \cellcolor{gray!6}{Desmatamento cr} & \cellcolor{gray!6}{Cs desordenado} & \cellcolor{gray!6}{1} & \cellcolor{gray!6}{17.9}\\
\cmidrule{1-5}
Cicatriz de queimada & Desmatamento cr & Degradacao & 38 & 443.8\\
\cmidrule{1-5}
\cellcolor{gray!6}{Cicatriz de queimada} & \cellcolor{gray!6}{Desmatamento cr} & \cellcolor{gray!6}{Desmatamento cr} & \cellcolor{gray!6}{196} & \cellcolor{gray!6}{2474.5}\\
\cmidrule{1-5}
Cicatriz de queimada & Desmatamento cr & Desmatamento veg & 10 & 170.9\\
\cmidrule{1-5}
\cellcolor{gray!6}{Cicatriz de queimada} & \cellcolor{gray!6}{Desmatamento veg} & \cellcolor{gray!6}{Cicatriz de queimada} & \cellcolor{gray!6}{7} & \cellcolor{gray!6}{64.4}\\
\cmidrule{1-5}
Cicatriz de queimada & Desmatamento veg & Degradacao & 2 & 9.8\\
\cmidrule{1-5}
\cellcolor{gray!6}{Cicatriz de queimada} & \cellcolor{gray!6}{Desmatamento veg} & \cellcolor{gray!6}{Desmatamento cr} & \cellcolor{gray!6}{14} & \cellcolor{gray!6}{177.2}\\
\cmidrule{1-5}
Cicatriz de queimada & Desmatamento veg & Desmatamento veg & 1 & 3.5\\
\cmidrule{1-5}
\cellcolor{gray!6}{Cicatriz de queimada} & \cellcolor{gray!6}{Mineracao} & \cellcolor{gray!6}{Mineracao} & \cellcolor{gray!6}{5} & \cellcolor{gray!6}{22.4}\\
\cmidrule{1-5}
Cs desordenado & Cicatriz de queimada & Cicatriz de queimada & 11 & 492.6\\
\cmidrule{1-5}
\cellcolor{gray!6}{Cs desordenado} & \cellcolor{gray!6}{Cicatriz de queimada} & \cellcolor{gray!6}{Cs desordenado} & \cellcolor{gray!6}{3} & \cellcolor{gray!6}{75.4}\\
\cmidrule{1-5}
Cs desordenado & Cicatriz de queimada & Desmatamento cr & 6 & 157.9\\
\cmidrule{1-5}
\cellcolor{gray!6}{Cs desordenado} & \cellcolor{gray!6}{Cs desordenado} & \cellcolor{gray!6}{Cicatriz de queimada} & \cellcolor{gray!6}{4} & \cellcolor{gray!6}{143.9}\\
\cmidrule{1-5}
Cs desordenado & Cs desordenado & Cs desordenado & 29 & 842.7\\
\cmidrule{1-5}
\cellcolor{gray!6}{Cs desordenado} & \cellcolor{gray!6}{Cs desordenado} & \cellcolor{gray!6}{Cs geometrico} & \cellcolor{gray!6}{1} & \cellcolor{gray!6}{4.3}\\
\cmidrule{1-5}
Cs desordenado & Cs desordenado & Degradacao & 6 & 133.0\\
\cmidrule{1-5}
\cellcolor{gray!6}{Cs desordenado} & \cellcolor{gray!6}{Cs desordenado} & \cellcolor{gray!6}{Desmatamento cr} & \cellcolor{gray!6}{2} & \cellcolor{gray!6}{11.3}\\
\cmidrule{1-5}
Cs desordenado & Cs geometrico & Cicatriz de queimada & 6 & 250.2\\
\cmidrule{1-5}
\cellcolor{gray!6}{Cs desordenado} & \cellcolor{gray!6}{Cs geometrico} & \cellcolor{gray!6}{Cs desordenado} & \cellcolor{gray!6}{7} & \cellcolor{gray!6}{303.6}\\
\cmidrule{1-5}
Cs desordenado & Cs geometrico & Cs geometrico & 4 & 675.3\\
\cmidrule{1-5}
\cellcolor{gray!6}{Cs desordenado} & \cellcolor{gray!6}{Cs geometrico} & \cellcolor{gray!6}{Degradacao} & \cellcolor{gray!6}{1} & \cellcolor{gray!6}{11.6}\\
\cmidrule{1-5}
Cs desordenado & Degradacao & Cicatriz de queimada & 7 & 52.2\\
\cmidrule{1-5}
\cellcolor{gray!6}{Cs desordenado} & \cellcolor{gray!6}{Degradacao} & \cellcolor{gray!6}{Cs desordenado} & \cellcolor{gray!6}{10} & \cellcolor{gray!6}{1002.9}\\
\cmidrule{1-5}
Cs desordenado & Degradacao & Cs geometrico & 2 & 11.1\\
\cmidrule{1-5}
\cellcolor{gray!6}{Cs desordenado} & \cellcolor{gray!6}{Degradacao} & \cellcolor{gray!6}{Degradacao} & \cellcolor{gray!6}{2} & \cellcolor{gray!6}{19.3}\\
\cmidrule{1-5}
Cs desordenado & Degradacao & Desmatamento cr & 5 & 39.8\\
\cmidrule{1-5}
\cellcolor{gray!6}{Cs desordenado} & \cellcolor{gray!6}{Desmatamento cr} & \cellcolor{gray!6}{Cicatriz de queimada} & \cellcolor{gray!6}{1} & \cellcolor{gray!6}{3.4}\\
\cmidrule{1-5}
Cs desordenado & Desmatamento cr & Desmatamento cr & 10 & 140.5\\
\cmidrule{1-5}
\cellcolor{gray!6}{Cs desordenado} & \cellcolor{gray!6}{Desmatamento cr} & \cellcolor{gray!6}{Desmatamento veg} & \cellcolor{gray!6}{2} & \cellcolor{gray!6}{11.0}\\
\cmidrule{1-5}
Cs desordenado & Desmatamento veg & Desmatamento cr & 1 & 5.1\\
\cmidrule{1-5}
\cellcolor{gray!6}{Cs desordenado} & \cellcolor{gray!6}{Desmatamento veg} & \cellcolor{gray!6}{Desmatamento veg} & \cellcolor{gray!6}{1} & \cellcolor{gray!6}{4.3}\\
\cmidrule{1-5}
Cs geometrico & Cicatriz de queimada & Cicatriz de queimada & 8 & 248.9\\
\cmidrule{1-5}
\cellcolor{gray!6}{Cs geometrico} & \cellcolor{gray!6}{Cicatriz de queimada} & \cellcolor{gray!6}{Cs geometrico} & \cellcolor{gray!6}{1} & \cellcolor{gray!6}{3.3}\\
\cmidrule{1-5}
Cs geometrico & Cicatriz de queimada & Degradacao & 7 & 49.2\\
\cmidrule{1-5}
\cellcolor{gray!6}{Cs geometrico} & \cellcolor{gray!6}{Cs desordenado} & \cellcolor{gray!6}{Cicatriz de queimada} & \cellcolor{gray!6}{4} & \cellcolor{gray!6}{135.9}\\
\cmidrule{1-5}
Cs geometrico & Cs desordenado & Cs desordenado & 4 & 139.0\\
\cmidrule{1-5}
\cellcolor{gray!6}{Cs geometrico} & \cellcolor{gray!6}{Cs desordenado} & \cellcolor{gray!6}{Cs geometrico} & \cellcolor{gray!6}{5} & \cellcolor{gray!6}{324.1}\\
\cmidrule{1-5}
Cs geometrico & Cs desordenado & Degradacao & 7 & 335.7\\
\cmidrule{1-5}
\cellcolor{gray!6}{Cs geometrico} & \cellcolor{gray!6}{Cs desordenado} & \cellcolor{gray!6}{Desmatamento cr} & \cellcolor{gray!6}{1} & \cellcolor{gray!6}{19.8}\\
\cmidrule{1-5}
Cs geometrico & Cs geometrico & Cicatriz de queimada & 20 & 878.2\\
\cmidrule{1-5}
\cellcolor{gray!6}{Cs geometrico} & \cellcolor{gray!6}{Cs geometrico} & \cellcolor{gray!6}{Cs desordenado} & \cellcolor{gray!6}{18} & \cellcolor{gray!6}{652.3}\\
\cmidrule{1-5}
Cs geometrico & Cs geometrico & Cs geometrico & 145 & 10345.3\\
\cmidrule{1-5}
\cellcolor{gray!6}{Cs geometrico} & \cellcolor{gray!6}{Cs geometrico} & \cellcolor{gray!6}{Degradacao} & \cellcolor{gray!6}{8} & \cellcolor{gray!6}{146.8}\\
\cmidrule{1-5}
Cs geometrico & Cs geometrico & Desmatamento cr & 41 & 1189.3\\
\cmidrule{1-5}
\cellcolor{gray!6}{Cs geometrico} & \cellcolor{gray!6}{Degradacao} & \cellcolor{gray!6}{Cicatriz de queimada} & \cellcolor{gray!6}{2} & \cellcolor{gray!6}{17.5}\\
\cmidrule{1-5}
Cs geometrico & Degradacao & Cs desordenado & 2 & 132.7\\
\cmidrule{1-5}
\cellcolor{gray!6}{Cs geometrico} & \cellcolor{gray!6}{Degradacao} & \cellcolor{gray!6}{Cs geometrico} & \cellcolor{gray!6}{31} & \cellcolor{gray!6}{2247.7}\\
\cmidrule{1-5}
Cs geometrico & Degradacao & Degradacao & 4 & 85.9\\
\cmidrule{1-5}
\cellcolor{gray!6}{Cs geometrico} & \cellcolor{gray!6}{Degradacao} & \cellcolor{gray!6}{Desmatamento cr} & \cellcolor{gray!6}{11} & \cellcolor{gray!6}{299.9}\\
\cmidrule{1-5}
Cs geometrico & Desmatamento cr & Cs desordenado & 2 & 9.3\\
\cmidrule{1-5}
\cellcolor{gray!6}{Cs geometrico} & \cellcolor{gray!6}{Desmatamento cr} & \cellcolor{gray!6}{Cs geometrico} & \cellcolor{gray!6}{1} & \cellcolor{gray!6}{4.5}\\
\cmidrule{1-5}
Cs geometrico & Desmatamento cr & Desmatamento cr & 2 & 10.4\\
\cmidrule{1-5}
\cellcolor{gray!6}{Cs geometrico} & \cellcolor{gray!6}{Desmatamento cr} & \cellcolor{gray!6}{Desmatamento veg} & \cellcolor{gray!6}{1} & \cellcolor{gray!6}{4.2}\\
\cmidrule{1-5}
Degradacao & Cicatriz de queimada & Cicatriz de queimada & 97 & 1851.6\\
\cmidrule{1-5}
\cellcolor{gray!6}{Degradacao} & \cellcolor{gray!6}{Cicatriz de queimada} & \cellcolor{gray!6}{Cs desordenado} & \cellcolor{gray!6}{15} & \cellcolor{gray!6}{351.2}\\
\cmidrule{1-5}
Degradacao & Cicatriz de queimada & Cs geometrico & 6 & 69.8\\
\cmidrule{1-5}
\cellcolor{gray!6}{Degradacao} & \cellcolor{gray!6}{Cicatriz de queimada} & \cellcolor{gray!6}{Degradacao} & \cellcolor{gray!6}{39} & \cellcolor{gray!6}{785.3}\\
\cmidrule{1-5}
Degradacao & Cicatriz de queimada & Desmatamento cr & 73 & 1041.0\\
\cmidrule{1-5}
\cellcolor{gray!6}{Degradacao} & \cellcolor{gray!6}{Cs desordenado} & \cellcolor{gray!6}{Cicatriz de queimada} & \cellcolor{gray!6}{15} & \cellcolor{gray!6}{278.1}\\
\cmidrule{1-5}
Degradacao & Cs desordenado & Cs desordenado & 33 & 901.4\\
\cmidrule{1-5}
\cellcolor{gray!6}{Degradacao} & \cellcolor{gray!6}{Cs desordenado} & \cellcolor{gray!6}{Cs geometrico} & \cellcolor{gray!6}{12} & \cellcolor{gray!6}{458.6}\\
\cmidrule{1-5}
Degradacao & Cs desordenado & Degradacao & 10 & 165.2\\
\cmidrule{1-5}
\cellcolor{gray!6}{Degradacao} & \cellcolor{gray!6}{Cs desordenado} & \cellcolor{gray!6}{Desmatamento cr} & \cellcolor{gray!6}{13} & \cellcolor{gray!6}{111.2}\\
\cmidrule{1-5}
Degradacao & Cs geometrico & Cicatriz de queimada & 30 & 832.2\\
\cmidrule{1-5}
\cellcolor{gray!6}{Degradacao} & \cellcolor{gray!6}{Cs geometrico} & \cellcolor{gray!6}{Cs desordenado} & \cellcolor{gray!6}{12} & \cellcolor{gray!6}{773.8}\\
\cmidrule{1-5}
Degradacao & Cs geometrico & Cs geometrico & 82 & 3016.7\\
\cmidrule{1-5}
\cellcolor{gray!6}{Degradacao} & \cellcolor{gray!6}{Cs geometrico} & \cellcolor{gray!6}{Degradacao} & \cellcolor{gray!6}{9} & \cellcolor{gray!6}{378.7}\\
\cmidrule{1-5}
Degradacao & Cs geometrico & Desmatamento cr & 43 & 837.1\\
\cmidrule{1-5}
\cellcolor{gray!6}{Degradacao} & \cellcolor{gray!6}{Degradacao} & \cellcolor{gray!6}{Cicatriz de queimada} & \cellcolor{gray!6}{54} & \cellcolor{gray!6}{1169.9}\\
\cmidrule{1-5}
Degradacao & Degradacao & Cs desordenado & 30 & 1445.8\\
\cmidrule{1-5}
\cellcolor{gray!6}{Degradacao} & \cellcolor{gray!6}{Degradacao} & \cellcolor{gray!6}{Cs geometrico} & \cellcolor{gray!6}{32} & \cellcolor{gray!6}{805.8}\\
\cmidrule{1-5}
Degradacao & Degradacao & Degradacao & 53 & 1178.8\\
\cmidrule{1-5}
\cellcolor{gray!6}{Degradacao} & \cellcolor{gray!6}{Degradacao} & \cellcolor{gray!6}{Desmatamento cr} & \cellcolor{gray!6}{121} & \cellcolor{gray!6}{1281.0}\\
\cmidrule{1-5}
Degradacao & Degradacao & Desmatamento veg & 4 & 33.7\\
\cmidrule{1-5}
\cellcolor{gray!6}{Degradacao} & \cellcolor{gray!6}{Desmatamento cr} & \cellcolor{gray!6}{Cicatriz de queimada} & \cellcolor{gray!6}{14} & \cellcolor{gray!6}{94.5}\\
\cmidrule{1-5}
Degradacao & Desmatamento cr & Cs desordenado & 1 & 4.6\\
\cmidrule{1-5}
\cellcolor{gray!6}{Degradacao} & \cellcolor{gray!6}{Desmatamento cr} & \cellcolor{gray!6}{Degradacao} & \cellcolor{gray!6}{48} & \cellcolor{gray!6}{668.3}\\
\cmidrule{1-5}
Degradacao & Desmatamento cr & Desmatamento cr & 169 & 1669.2\\
\cmidrule{1-5}
\cellcolor{gray!6}{Degradacao} & \cellcolor{gray!6}{Desmatamento cr} & \cellcolor{gray!6}{Desmatamento veg} & \cellcolor{gray!6}{13} & \cellcolor{gray!6}{225.8}\\
\cmidrule{1-5}
Degradacao & Desmatamento veg & Cicatriz de queimada & 13 & 208.0\\
\cmidrule{1-5}
\cellcolor{gray!6}{Degradacao} & \cellcolor{gray!6}{Desmatamento veg} & \cellcolor{gray!6}{Degradacao} & \cellcolor{gray!6}{2} & \cellcolor{gray!6}{19.3}\\
\cmidrule{1-5}
Degradacao & Desmatamento veg & Desmatamento cr & 25 & 258.3\\
\cmidrule{1-5}
\cellcolor{gray!6}{Degradacao} & \cellcolor{gray!6}{Desmatamento veg} & \cellcolor{gray!6}{Desmatamento veg} & \cellcolor{gray!6}{9} & \cellcolor{gray!6}{145.6}\\
\cmidrule{1-5}
Degradacao & Mineracao & Desmatamento cr & 1 & 4.3\\
\cmidrule{1-5}
\cellcolor{gray!6}{Desmatamento cr} & \cellcolor{gray!6}{Cicatriz de queimada} & \cellcolor{gray!6}{Cicatriz de queimada} & \cellcolor{gray!6}{53} & \cellcolor{gray!6}{1322.2}\\
\cmidrule{1-5}
Desmatamento cr & Cicatriz de queimada & Cs desordenado & 1 & 4.7\\
\cmidrule{1-5}
\cellcolor{gray!6}{Desmatamento cr} & \cellcolor{gray!6}{Cicatriz de queimada} & \cellcolor{gray!6}{Degradacao} & \cellcolor{gray!6}{24} & \cellcolor{gray!6}{261.8}\\
\cmidrule{1-5}
Desmatamento cr & Cicatriz de queimada & Desmatamento cr & 53 & 799.8\\
\cmidrule{1-5}
\cellcolor{gray!6}{Desmatamento cr} & \cellcolor{gray!6}{Cs desordenado} & \cellcolor{gray!6}{Cicatriz de queimada} & \cellcolor{gray!6}{1} & \cellcolor{gray!6}{5.6}\\
\cmidrule{1-5}
Desmatamento cr & Cs desordenado & Cs desordenado & 5 & 186.9\\
\cmidrule{1-5}
\cellcolor{gray!6}{Desmatamento cr} & \cellcolor{gray!6}{Cs desordenado} & \cellcolor{gray!6}{Degradacao} & \cellcolor{gray!6}{3} & \cellcolor{gray!6}{88.6}\\
\cmidrule{1-5}
Desmatamento cr & Cs desordenado & Desmatamento cr & 2 & 8.4\\
\cmidrule{1-5}
\cellcolor{gray!6}{Desmatamento cr} & \cellcolor{gray!6}{Cs desordenado} & \cellcolor{gray!6}{Desmatamento veg} & \cellcolor{gray!6}{3} & \cellcolor{gray!6}{49.1}\\
\cmidrule{1-5}
Desmatamento cr & Cs geometrico & Cs geometrico & 4 & 45.4\\
\cmidrule{1-5}
\cellcolor{gray!6}{Desmatamento cr} & \cellcolor{gray!6}{Degradacao} & \cellcolor{gray!6}{Cicatriz de queimada} & \cellcolor{gray!6}{42} & \cellcolor{gray!6}{593.7}\\
\cmidrule{1-5}
Desmatamento cr & Degradacao & Degradacao & 27 & 414.2\\
\cmidrule{1-5}
\cellcolor{gray!6}{Desmatamento cr} & \cellcolor{gray!6}{Degradacao} & \cellcolor{gray!6}{Desmatamento cr} & \cellcolor{gray!6}{141} & \cellcolor{gray!6}{1415.1}\\
\cmidrule{1-5}
Desmatamento cr & Degradacao & Desmatamento veg & 3 & 13.2\\
\cmidrule{1-5}
\cellcolor{gray!6}{Desmatamento cr} & \cellcolor{gray!6}{Desmatamento cr} & \cellcolor{gray!6}{Cicatriz de queimada} & \cellcolor{gray!6}{15} & \cellcolor{gray!6}{109.9}\\
\cmidrule{1-5}
Desmatamento cr & Desmatamento cr & Degradacao & 16 & 188.0\\
\cmidrule{1-5}
\cellcolor{gray!6}{Desmatamento cr} & \cellcolor{gray!6}{Desmatamento cr} & \cellcolor{gray!6}{Desmatamento cr} & \cellcolor{gray!6}{258} & \cellcolor{gray!6}{1889.4}\\
\cmidrule{1-5}
Desmatamento cr & Desmatamento cr & Desmatamento veg & 16 & 136.5\\
\cmidrule{1-5}
\cellcolor{gray!6}{Desmatamento cr} & \cellcolor{gray!6}{Desmatamento cr} & \cellcolor{gray!6}{Mineracao} & \cellcolor{gray!6}{1} & \cellcolor{gray!6}{6.3}\\
\cmidrule{1-5}
Desmatamento cr & Desmatamento veg & Cicatriz de queimada & 10 & 90.4\\
\cmidrule{1-5}
\cellcolor{gray!6}{Desmatamento cr} & \cellcolor{gray!6}{Desmatamento veg} & \cellcolor{gray!6}{Degradacao} & \cellcolor{gray!6}{7} & \cellcolor{gray!6}{143.1}\\
\cmidrule{1-5}
Desmatamento cr & Desmatamento veg & Desmatamento cr & 52 & 481.2\\
\cmidrule{1-5}
\cellcolor{gray!6}{Desmatamento cr} & \cellcolor{gray!6}{Desmatamento veg} & \cellcolor{gray!6}{Desmatamento veg} & \cellcolor{gray!6}{23} & \cellcolor{gray!6}{274.5}\\
\cmidrule{1-5}
Desmatamento veg & Cicatriz de queimada & Cicatriz de queimada & 2 & 12.5\\
\cmidrule{1-5}
\cellcolor{gray!6}{Desmatamento veg} & \cellcolor{gray!6}{Cicatriz de queimada} & \cellcolor{gray!6}{Degradacao} & \cellcolor{gray!6}{3} & \cellcolor{gray!6}{44.0}\\
\cmidrule{1-5}
Desmatamento veg & Cicatriz de queimada & Desmatamento cr & 2 & 29.7\\
\cmidrule{1-5}
\cellcolor{gray!6}{Desmatamento veg} & \cellcolor{gray!6}{Cicatriz de queimada} & \cellcolor{gray!6}{Desmatamento veg} & \cellcolor{gray!6}{1} & \cellcolor{gray!6}{64.6}\\
\cmidrule{1-5}
Desmatamento veg & Degradacao & Cicatriz de queimada & 1 & 9.5\\
\cmidrule{1-5}
\cellcolor{gray!6}{Desmatamento veg} & \cellcolor{gray!6}{Degradacao} & \cellcolor{gray!6}{Desmatamento cr} & \cellcolor{gray!6}{3} & \cellcolor{gray!6}{19.3}\\
\cmidrule{1-5}
Desmatamento veg & Desmatamento cr & Degradacao & 1 & 11.8\\
\cmidrule{1-5}
\cellcolor{gray!6}{Desmatamento veg} & \cellcolor{gray!6}{Desmatamento cr} & \cellcolor{gray!6}{Desmatamento cr} & \cellcolor{gray!6}{6} & \cellcolor{gray!6}{32.6}\\
\cmidrule{1-5}
Desmatamento veg & Desmatamento veg & Desmatamento cr & 7 & 113.2\\
\cmidrule{1-5}
\cellcolor{gray!6}{Desmatamento veg} & \cellcolor{gray!6}{Desmatamento veg} & \cellcolor{gray!6}{Desmatamento veg} & \cellcolor{gray!6}{3} & \cellcolor{gray!6}{19.3}\\
\cmidrule{1-5}
Mineracao & Mineracao & Degradacao & 1 & 4.2\\
\cmidrule{1-5}
\cellcolor{gray!6}{Mineracao} & \cellcolor{gray!6}{Mineracao} & \cellcolor{gray!6}{Mineracao} & \cellcolor{gray!6}{19} & \cellcolor{gray!6}{104.1}\\
\bottomrule
\end{longtabu}
\endgroup{}
\end{frame}

\begin{frame}
    \frametitle{DETER subareas (4 warnings)}
    \begin{figure}[h] 
        \includegraphics[width=0.65\linewidth]
        {./figures/plot_deter_subarea_trajectory_4.png}
        \caption{Tajectory of subareas with 4 wanings.}
        \label{fig:deter_subarea_trajectory_4}
    \end{figure}
\end{frame}

\begin{frame}[allowframebreaks]
    \frametitle{DETER - Top 5 trajectories (4 warnings)}
    \begingroup\fontsize{7}{9}\selectfont

\begin{longtabu} to \linewidth {>{\raggedright}X>{\raggedright}X>{\raggedright}X>{\raggedright}X>{\raggedleft}X>{\raggedleft}X}
\toprule
position\_1 & position\_2 & position\_3 & position\_4 & n\_traj & subarea\_ha\\
\midrule
\cellcolor{gray!6}{Cicatriz de queimada} & \cellcolor{gray!6}{Cicatriz de queimada} & \cellcolor{gray!6}{Cicatriz de queimada} & \cellcolor{gray!6}{Cicatriz de queimada} & \cellcolor{gray!6}{92} & \cellcolor{gray!6}{3003.6}\\
\cmidrule{1-6}
Cicatriz de queimada & Cicatriz de queimada & Cicatriz de queimada & Degradacao & 12 & 725.4\\
\cmidrule{1-6}
\cellcolor{gray!6}{Cicatriz de queimada} & \cellcolor{gray!6}{Cicatriz de queimada} & \cellcolor{gray!6}{Cicatriz de queimada} & \cellcolor{gray!6}{Desmatamento cr} & \cellcolor{gray!6}{14} & \cellcolor{gray!6}{147.2}\\
\cmidrule{1-6}
Cicatriz de queimada & Cicatriz de queimada & Degradacao & Cicatriz de queimada & 5 & 257.0\\
\cmidrule{1-6}
\cellcolor{gray!6}{Cicatriz de queimada} & \cellcolor{gray!6}{Cicatriz de queimada} & \cellcolor{gray!6}{Degradacao} & \cellcolor{gray!6}{Cs desordenado} & \cellcolor{gray!6}{7} & \cellcolor{gray!6}{184.8}\\
\cmidrule{1-6}
Cicatriz de queimada & Cicatriz de queimada & Degradacao & Degradacao & 4 & 128.2\\
\cmidrule{1-6}
\cellcolor{gray!6}{Cicatriz de queimada} & \cellcolor{gray!6}{Cicatriz de queimada} & \cellcolor{gray!6}{Degradacao} & \cellcolor{gray!6}{Desmatamento cr} & \cellcolor{gray!6}{8} & \cellcolor{gray!6}{466.3}\\
\cmidrule{1-6}
Cicatriz de queimada & Cicatriz de queimada & Desmatamento cr & Cicatriz de queimada & 3 & 26.4\\
\cmidrule{1-6}
\cellcolor{gray!6}{Cicatriz de queimada} & \cellcolor{gray!6}{Cicatriz de queimada} & \cellcolor{gray!6}{Desmatamento cr} & \cellcolor{gray!6}{Degradacao} & \cellcolor{gray!6}{10} & \cellcolor{gray!6}{525.2}\\
\cmidrule{1-6}
Cicatriz de queimada & Cicatriz de queimada & Desmatamento cr & Desmatamento cr & 9 & 68.2\\
\cmidrule{1-6}
\cellcolor{gray!6}{Cicatriz de queimada} & \cellcolor{gray!6}{Cs desordenado} & \cellcolor{gray!6}{Cicatriz de queimada} & \cellcolor{gray!6}{Degradacao} & \cellcolor{gray!6}{5} & \cellcolor{gray!6}{81.7}\\
\cmidrule{1-6}
Cicatriz de queimada & Cs desordenado & Cicatriz de queimada & Desmatamento cr & 1 & 13.9\\
\cmidrule{1-6}
\cellcolor{gray!6}{Cicatriz de queimada} & \cellcolor{gray!6}{Cs desordenado} & \cellcolor{gray!6}{Degradacao} & \cellcolor{gray!6}{Degradacao} & \cellcolor{gray!6}{1} & \cellcolor{gray!6}{28.7}\\
\cmidrule{1-6}
Cicatriz de queimada & Cs geometrico & Cs geometrico & Cicatriz de queimada & 1 & 6.5\\
\cmidrule{1-6}
\cellcolor{gray!6}{Cicatriz de queimada} & \cellcolor{gray!6}{Cs geometrico} & \cellcolor{gray!6}{Degradacao} & \cellcolor{gray!6}{Degradacao} & \cellcolor{gray!6}{5} & \cellcolor{gray!6}{197.0}\\
\cmidrule{1-6}
Cicatriz de queimada & Cs geometrico & Desmatamento cr & Degradacao & 2 & 71.0\\
\cmidrule{1-6}
\cellcolor{gray!6}{Cicatriz de queimada} & \cellcolor{gray!6}{Degradacao} & \cellcolor{gray!6}{Cicatriz de queimada} & \cellcolor{gray!6}{Cicatriz de queimada} & \cellcolor{gray!6}{2} & \cellcolor{gray!6}{13.2}\\
\cmidrule{1-6}
Cicatriz de queimada & Degradacao & Cicatriz de queimada & Cs geometrico & 1 & 71.6\\
\cmidrule{1-6}
\cellcolor{gray!6}{Cicatriz de queimada} & \cellcolor{gray!6}{Degradacao} & \cellcolor{gray!6}{Cicatriz de queimada} & \cellcolor{gray!6}{Degradacao} & \cellcolor{gray!6}{1} & \cellcolor{gray!6}{3.7}\\
\cmidrule{1-6}
Cicatriz de queimada & Degradacao & Cicatriz de queimada & Desmatamento cr & 1 & 3.4\\
\cmidrule{1-6}
\cellcolor{gray!6}{Cicatriz de queimada} & \cellcolor{gray!6}{Degradacao} & \cellcolor{gray!6}{Cs desordenado} & \cellcolor{gray!6}{Degradacao} & \cellcolor{gray!6}{1} & \cellcolor{gray!6}{34.0}\\
\cmidrule{1-6}
Cicatriz de queimada & Degradacao & Degradacao & Cicatriz de queimada & 21 & 431.0\\
\cmidrule{1-6}
\cellcolor{gray!6}{Cicatriz de queimada} & \cellcolor{gray!6}{Degradacao} & \cellcolor{gray!6}{Degradacao} & \cellcolor{gray!6}{Degradacao} & \cellcolor{gray!6}{2} & \cellcolor{gray!6}{22.5}\\
\cmidrule{1-6}
Cicatriz de queimada & Degradacao & Degradacao & Desmatamento cr & 4 & 25.8\\
\cmidrule{1-6}
\cellcolor{gray!6}{Cicatriz de queimada} & \cellcolor{gray!6}{Degradacao} & \cellcolor{gray!6}{Desmatamento cr} & \cellcolor{gray!6}{Degradacao} & \cellcolor{gray!6}{4} & \cellcolor{gray!6}{62.0}\\
\cmidrule{1-6}
Cicatriz de queimada & Degradacao & Desmatamento cr & Desmatamento cr & 6 & 63.4\\
\cmidrule{1-6}
\cellcolor{gray!6}{Cicatriz de queimada} & \cellcolor{gray!6}{Degradacao} & \cellcolor{gray!6}{Desmatamento veg} & \cellcolor{gray!6}{Desmatamento cr} & \cellcolor{gray!6}{1} & \cellcolor{gray!6}{5.4}\\
\cmidrule{1-6}
Cicatriz de queimada & Desmatamento cr & Cicatriz de queimada & Cicatriz de queimada & 9 & 390.8\\
\cmidrule{1-6}
\cellcolor{gray!6}{Cicatriz de queimada} & \cellcolor{gray!6}{Desmatamento cr} & \cellcolor{gray!6}{Cicatriz de queimada} & \cellcolor{gray!6}{Desmatamento cr} & \cellcolor{gray!6}{1} & \cellcolor{gray!6}{6.1}\\
\cmidrule{1-6}
Cicatriz de queimada & Desmatamento cr & Desmatamento cr & Cicatriz de queimada & 1 & 5.3\\
\cmidrule{1-6}
\cellcolor{gray!6}{Cicatriz de queimada} & \cellcolor{gray!6}{Desmatamento cr} & \cellcolor{gray!6}{Desmatamento cr} & \cellcolor{gray!6}{Desmatamento cr} & \cellcolor{gray!6}{5} & \cellcolor{gray!6}{43.6}\\
\cmidrule{1-6}
Cicatriz de queimada & Desmatamento cr & Desmatamento veg & Desmatamento cr & 1 & 4.7\\
\cmidrule{1-6}
\cellcolor{gray!6}{Cs desordenado} & \cellcolor{gray!6}{Cicatriz de queimada} & \cellcolor{gray!6}{Cs desordenado} & \cellcolor{gray!6}{Cs desordenado} & \cellcolor{gray!6}{1} & \cellcolor{gray!6}{38.9}\\
\cmidrule{1-6}
Cs desordenado & Cs desordenado & Cs desordenado & Cs desordenado & 1 & 13.9\\
\cmidrule{1-6}
\cellcolor{gray!6}{Cs desordenado} & \cellcolor{gray!6}{Cs desordenado} & \cellcolor{gray!6}{Cs geometrico} & \cellcolor{gray!6}{Cs desordenado} & \cellcolor{gray!6}{1} & \cellcolor{gray!6}{17.7}\\
\cmidrule{1-6}
Cs desordenado & Cs geometrico & Cs geometrico & Cs desordenado & 3 & 22.4\\
\cmidrule{1-6}
\cellcolor{gray!6}{Cs geometrico} & \cellcolor{gray!6}{Cicatriz de queimada} & \cellcolor{gray!6}{Cs geometrico} & \cellcolor{gray!6}{Cs geometrico} & \cellcolor{gray!6}{11} & \cellcolor{gray!6}{238.4}\\
\cmidrule{1-6}
Cs geometrico & Cs desordenado & Cs geometrico & Cs geometrico & 1 & 32.4\\
\cmidrule{1-6}
\cellcolor{gray!6}{Cs geometrico} & \cellcolor{gray!6}{Cs desordenado} & \cellcolor{gray!6}{Degradacao} & \cellcolor{gray!6}{Cs desordenado} & \cellcolor{gray!6}{2} & \cellcolor{gray!6}{58.7}\\
\cmidrule{1-6}
Cs geometrico & Cs geometrico & Cs desordenado & Cicatriz de queimada & 2 & 118.2\\
\cmidrule{1-6}
\cellcolor{gray!6}{Cs geometrico} & \cellcolor{gray!6}{Cs geometrico} & \cellcolor{gray!6}{Cs geometrico} & \cellcolor{gray!6}{Cs geometrico} & \cellcolor{gray!6}{16} & \cellcolor{gray!6}{892.2}\\
\cmidrule{1-6}
Cs geometrico & Cs geometrico & Cs geometrico & Degradacao & 3 & 178.6\\
\cmidrule{1-6}
\cellcolor{gray!6}{Cs geometrico} & \cellcolor{gray!6}{Cs geometrico} & \cellcolor{gray!6}{Cs geometrico} & \cellcolor{gray!6}{Desmatamento cr} & \cellcolor{gray!6}{1} & \cellcolor{gray!6}{7.3}\\
\cmidrule{1-6}
Cs geometrico & Cs geometrico & Desmatamento cr & Cs geometrico & 1 & 3.7\\
\cmidrule{1-6}
\cellcolor{gray!6}{Cs geometrico} & \cellcolor{gray!6}{Degradacao} & \cellcolor{gray!6}{Cs geometrico} & \cellcolor{gray!6}{Cicatriz de queimada} & \cellcolor{gray!6}{2} & \cellcolor{gray!6}{14.0}\\
\cmidrule{1-6}
Cs geometrico & Degradacao & Cs geometrico & Cs desordenado & 2 & 46.1\\
\cmidrule{1-6}
\cellcolor{gray!6}{Cs geometrico} & \cellcolor{gray!6}{Degradacao} & \cellcolor{gray!6}{Cs geometrico} & \cellcolor{gray!6}{Cs geometrico} & \cellcolor{gray!6}{5} & \cellcolor{gray!6}{51.5}\\
\cmidrule{1-6}
Cs geometrico & Desmatamento cr & Cs geometrico & Cs geometrico & 1 & 218.8\\
\cmidrule{1-6}
\cellcolor{gray!6}{Degradacao} & \cellcolor{gray!6}{Cicatriz de queimada} & \cellcolor{gray!6}{Cicatriz de queimada} & \cellcolor{gray!6}{Cicatriz de queimada} & \cellcolor{gray!6}{1} & \cellcolor{gray!6}{6.0}\\
\cmidrule{1-6}
Degradacao & Cicatriz de queimada & Cicatriz de queimada & Desmatamento cr & 2 & 32.6\\
\cmidrule{1-6}
\cellcolor{gray!6}{Degradacao} & \cellcolor{gray!6}{Cicatriz de queimada} & \cellcolor{gray!6}{Cs desordenado} & \cellcolor{gray!6}{Cicatriz de queimada} & \cellcolor{gray!6}{2} & \cellcolor{gray!6}{39.7}\\
\cmidrule{1-6}
Degradacao & Cicatriz de queimada & Cs desordenado & Degradacao & 1 & 22.3\\
\cmidrule{1-6}
\cellcolor{gray!6}{Degradacao} & \cellcolor{gray!6}{Cicatriz de queimada} & \cellcolor{gray!6}{Cs desordenado} & \cellcolor{gray!6}{Desmatamento cr} & \cellcolor{gray!6}{1} & \cellcolor{gray!6}{20.0}\\
\cmidrule{1-6}
Degradacao & Cicatriz de queimada & Cs geometrico & Cs geometrico & 23 & 373.2\\
\cmidrule{1-6}
\cellcolor{gray!6}{Degradacao} & \cellcolor{gray!6}{Cicatriz de queimada} & \cellcolor{gray!6}{Degradacao} & \cellcolor{gray!6}{Cicatriz de queimada} & \cellcolor{gray!6}{3} & \cellcolor{gray!6}{224.5}\\
\cmidrule{1-6}
Degradacao & Cicatriz de queimada & Degradacao & Cs geometrico & 12 & 171.1\\
\cmidrule{1-6}
\cellcolor{gray!6}{Degradacao} & \cellcolor{gray!6}{Cicatriz de queimada} & \cellcolor{gray!6}{Desmatamento cr} & \cellcolor{gray!6}{Degradacao} & \cellcolor{gray!6}{1} & \cellcolor{gray!6}{10.4}\\
\cmidrule{1-6}
Degradacao & Cs desordenado & Degradacao & Cs desordenado & 1 & 33.6\\
\cmidrule{1-6}
\cellcolor{gray!6}{Degradacao} & \cellcolor{gray!6}{Cs geometrico} & \cellcolor{gray!6}{Cs geometrico} & \cellcolor{gray!6}{Cicatriz de queimada} & \cellcolor{gray!6}{7} & \cellcolor{gray!6}{387.4}\\
\cmidrule{1-6}
Degradacao & Cs geometrico & Cs geometrico & Cs desordenado & 1 & 3.3\\
\cmidrule{1-6}
\cellcolor{gray!6}{Degradacao} & \cellcolor{gray!6}{Cs geometrico} & \cellcolor{gray!6}{Cs geometrico} & \cellcolor{gray!6}{Cs geometrico} & \cellcolor{gray!6}{16} & \cellcolor{gray!6}{515.6}\\
\cmidrule{1-6}
Degradacao & Cs geometrico & Cs geometrico & Degradacao & 1 & 8.2\\
\cmidrule{1-6}
\cellcolor{gray!6}{Degradacao} & \cellcolor{gray!6}{Cs geometrico} & \cellcolor{gray!6}{Desmatamento cr} & \cellcolor{gray!6}{Degradacao} & \cellcolor{gray!6}{1} & \cellcolor{gray!6}{17.1}\\
\cmidrule{1-6}
Degradacao & Cs geometrico & Desmatamento cr & Desmatamento cr & 1 & 3.0\\
\cmidrule{1-6}
\cellcolor{gray!6}{Degradacao} & \cellcolor{gray!6}{Degradacao} & \cellcolor{gray!6}{Cs desordenado} & \cellcolor{gray!6}{Cs desordenado} & \cellcolor{gray!6}{1} & \cellcolor{gray!6}{4.4}\\
\cmidrule{1-6}
Degradacao & Degradacao & Cs geometrico & Cs geometrico & 4 & 26.0\\
\cmidrule{1-6}
\cellcolor{gray!6}{Degradacao} & \cellcolor{gray!6}{Degradacao} & \cellcolor{gray!6}{Cs geometrico} & \cellcolor{gray!6}{Degradacao} & \cellcolor{gray!6}{1} & \cellcolor{gray!6}{4.9}\\
\cmidrule{1-6}
Degradacao & Degradacao & Degradacao & Degradacao & 1 & 12.0\\
\cmidrule{1-6}
\cellcolor{gray!6}{Degradacao} & \cellcolor{gray!6}{Degradacao} & \cellcolor{gray!6}{Degradacao} & \cellcolor{gray!6}{Desmatamento cr} & \cellcolor{gray!6}{4} & \cellcolor{gray!6}{50.2}\\
\cmidrule{1-6}
Degradacao & Degradacao & Desmatamento cr & Degradacao & 5 & 51.0\\
\cmidrule{1-6}
\cellcolor{gray!6}{Degradacao} & \cellcolor{gray!6}{Degradacao} & \cellcolor{gray!6}{Desmatamento cr} & \cellcolor{gray!6}{Desmatamento cr} & \cellcolor{gray!6}{3} & \cellcolor{gray!6}{17.0}\\
\cmidrule{1-6}
Degradacao & Desmatamento cr & Cicatriz de queimada & Cicatriz de queimada & 2 & 9.4\\
\cmidrule{1-6}
\cellcolor{gray!6}{Degradacao} & \cellcolor{gray!6}{Desmatamento cr} & \cellcolor{gray!6}{Cicatriz de queimada} & \cellcolor{gray!6}{Desmatamento cr} & \cellcolor{gray!6}{1} & \cellcolor{gray!6}{22.5}\\
\cmidrule{1-6}
Degradacao & Desmatamento cr & Cs geometrico & Cs geometrico & 1 & 96.7\\
\cmidrule{1-6}
\cellcolor{gray!6}{Degradacao} & \cellcolor{gray!6}{Desmatamento cr} & \cellcolor{gray!6}{Degradacao} & \cellcolor{gray!6}{Desmatamento cr} & \cellcolor{gray!6}{1} & \cellcolor{gray!6}{3.5}\\
\cmidrule{1-6}
Degradacao & Desmatamento cr & Desmatamento cr & Degradacao & 9 & 181.3\\
\cmidrule{1-6}
\cellcolor{gray!6}{Degradacao} & \cellcolor{gray!6}{Desmatamento cr} & \cellcolor{gray!6}{Desmatamento cr} & \cellcolor{gray!6}{Desmatamento cr} & \cellcolor{gray!6}{2} & \cellcolor{gray!6}{11.2}\\
\cmidrule{1-6}
Degradacao & Desmatamento cr & Desmatamento veg & Desmatamento cr & 1 & 3.5\\
\cmidrule{1-6}
\cellcolor{gray!6}{Degradacao} & \cellcolor{gray!6}{Desmatamento veg} & \cellcolor{gray!6}{Degradacao} & \cellcolor{gray!6}{Desmatamento cr} & \cellcolor{gray!6}{2} & \cellcolor{gray!6}{29.8}\\
\cmidrule{1-6}
Desmatamento cr & Cicatriz de queimada & Cicatriz de queimada & Cicatriz de queimada & 2 & 18.9\\
\cmidrule{1-6}
\cellcolor{gray!6}{Desmatamento cr} & \cellcolor{gray!6}{Cicatriz de queimada} & \cellcolor{gray!6}{Cicatriz de queimada} & \cellcolor{gray!6}{Cs desordenado} & \cellcolor{gray!6}{1} & \cellcolor{gray!6}{3.3}\\
\cmidrule{1-6}
Desmatamento cr & Cicatriz de queimada & Cicatriz de queimada & Degradacao & 1 & 4.5\\
\cmidrule{1-6}
\cellcolor{gray!6}{Desmatamento cr} & \cellcolor{gray!6}{Cicatriz de queimada} & \cellcolor{gray!6}{Degradacao} & \cellcolor{gray!6}{Degradacao} & \cellcolor{gray!6}{1} & \cellcolor{gray!6}{8.0}\\
\cmidrule{1-6}
Desmatamento cr & Cicatriz de queimada & Desmatamento cr & Cicatriz de queimada & 6 & 184.2\\
\cmidrule{1-6}
\cellcolor{gray!6}{Desmatamento cr} & \cellcolor{gray!6}{Cicatriz de queimada} & \cellcolor{gray!6}{Desmatamento cr} & \cellcolor{gray!6}{Desmatamento cr} & \cellcolor{gray!6}{3} & \cellcolor{gray!6}{28.8}\\
\cmidrule{1-6}
Desmatamento cr & Cs desordenado & Degradacao & Cs desordenado & 1 & 48.8\\
\cmidrule{1-6}
\cellcolor{gray!6}{Desmatamento cr} & \cellcolor{gray!6}{Cs desordenado} & \cellcolor{gray!6}{Desmatamento veg} & \cellcolor{gray!6}{Desmatamento cr} & \cellcolor{gray!6}{1} & \cellcolor{gray!6}{8.4}\\
\cmidrule{1-6}
Desmatamento cr & Cs desordenado & Desmatamento veg & Desmatamento veg & 2 & 43.4\\
\cmidrule{1-6}
\cellcolor{gray!6}{Desmatamento cr} & \cellcolor{gray!6}{Degradacao} & \cellcolor{gray!6}{Cicatriz de queimada} & \cellcolor{gray!6}{Desmatamento cr} & \cellcolor{gray!6}{1} & \cellcolor{gray!6}{16.7}\\
\cmidrule{1-6}
Desmatamento cr & Degradacao & Degradacao & Cicatriz de queimada & 1 & 215.5\\
\cmidrule{1-6}
\cellcolor{gray!6}{Desmatamento cr} & \cellcolor{gray!6}{Degradacao} & \cellcolor{gray!6}{Desmatamento cr} & \cellcolor{gray!6}{Degradacao} & \cellcolor{gray!6}{2} & \cellcolor{gray!6}{29.7}\\
\cmidrule{1-6}
Desmatamento cr & Degradacao & Desmatamento cr & Desmatamento cr & 5 & 71.9\\
\cmidrule{1-6}
\cellcolor{gray!6}{Desmatamento cr} & \cellcolor{gray!6}{Degradacao} & \cellcolor{gray!6}{Desmatamento veg} & \cellcolor{gray!6}{Desmatamento cr} & \cellcolor{gray!6}{1} & \cellcolor{gray!6}{6.1}\\
\cmidrule{1-6}
Desmatamento cr & Desmatamento cr & Cicatriz de queimada & Desmatamento cr & 1 & 5.0\\
\cmidrule{1-6}
\cellcolor{gray!6}{Desmatamento cr} & \cellcolor{gray!6}{Desmatamento cr} & \cellcolor{gray!6}{Degradacao} & \cellcolor{gray!6}{Cicatriz de queimada} & \cellcolor{gray!6}{1} & \cellcolor{gray!6}{5.1}\\
\cmidrule{1-6}
Desmatamento cr & Desmatamento cr & Desmatamento cr & Desmatamento cr & 4 & 26.3\\
\cmidrule{1-6}
\cellcolor{gray!6}{Desmatamento cr} & \cellcolor{gray!6}{Desmatamento cr} & \cellcolor{gray!6}{Desmatamento cr} & \cellcolor{gray!6}{Desmatamento veg} & \cellcolor{gray!6}{1} & \cellcolor{gray!6}{13.1}\\
\cmidrule{1-6}
Desmatamento cr & Desmatamento cr & Desmatamento veg & Desmatamento cr & 1 & 4.4\\
\cmidrule{1-6}
\cellcolor{gray!6}{Desmatamento cr} & \cellcolor{gray!6}{Desmatamento veg} & \cellcolor{gray!6}{Degradacao} & \cellcolor{gray!6}{Desmatamento cr} & \cellcolor{gray!6}{5} & \cellcolor{gray!6}{71.8}\\
\cmidrule{1-6}
Desmatamento cr & Desmatamento veg & Desmatamento cr & Desmatamento cr & 1 & 11.4\\
\cmidrule{1-6}
\cellcolor{gray!6}{Desmatamento cr} & \cellcolor{gray!6}{Desmatamento veg} & \cellcolor{gray!6}{Desmatamento veg} & \cellcolor{gray!6}{Desmatamento veg} & \cellcolor{gray!6}{1} & \cellcolor{gray!6}{4.9}\\
\cmidrule{1-6}
Mineracao & Mineracao & Mineracao & Mineracao & 1 & 4.2\\
\bottomrule
\end{longtabu}
\endgroup{}
\end{frame}

\begin{frame}
    \frametitle{DETER subareas (5 warnings)}
    \begin{figure}[h] 
        \includegraphics[width=0.65\linewidth]
        {./figures/plot_deter_subarea_trajectory_5.png}
        \caption{Tajectory of subareas with 5 wanings.}
        \label{fig:deter_subarea_trajectory_5}
    \end{figure}
\end{frame}

\begin{frame}[allowframebreaks]
    \frametitle{DETER - Top 5 trajectories (5 warnings)}
    \begingroup\fontsize{7}{9}\selectfont

\begin{longtabu} to \linewidth {>{\raggedright}X>{\raggedright}X>{\raggedright}X>{\raggedright}X>{\raggedright}X>{\raggedleft}X>{\raggedleft}X}
\toprule
position\_1 & position\_2 & position\_3 & position\_4 & position\_5 & area\_ha & n\_traj\\
\midrule
\cellcolor{gray!6}{Cicatriz de queimada} & \cellcolor{gray!6}{Cicatriz de queimada} & \cellcolor{gray!6}{Cicatriz de queimada} & \cellcolor{gray!6}{Cicatriz de queimada} & \cellcolor{gray!6}{Cicatriz de queimada} & \cellcolor{gray!6}{289.7} & \cellcolor{gray!6}{11}\\
\cmidrule{1-7}
Degradacao & Cs geometrico & Cs geometrico & Cs geometrico & Degradacao & 80.5 & 2\\
\cmidrule{1-7}
\cellcolor{gray!6}{Cs geometrico} & \cellcolor{gray!6}{Degradacao} & \cellcolor{gray!6}{Cs geometrico} & \cellcolor{gray!6}{Cs geometrico} & \cellcolor{gray!6}{Cs geometrico} & \cellcolor{gray!6}{32.9} & \cellcolor{gray!6}{3}\\
\cmidrule{1-7}
Cs geometrico & Cs geometrico & Cs geometrico & Cs geometrico & Cs geometrico & 6.5 & 2\\
\cmidrule{1-7}
\cellcolor{gray!6}{Total} & \cellcolor{gray!6}{-} & \cellcolor{gray!6}{-} & \cellcolor{gray!6}{-} & \cellcolor{gray!6}{-} & \cellcolor{gray!6}{409.7} & \cellcolor{gray!6}{18}\\
\bottomrule
\end{longtabu}
\endgroup{}
\end{frame}



\subsection{Trajectories (DETER \& PRODES)} 

\begin{frame}
    \frametitle{DETER \& PRODES subareas (2 warnings)}
    \begin{figure}[h] 
        \includegraphics[width=0.65\linewidth]
        {./figures/plot_deter_prodes_subarea_trajectory_2.png}
        \caption{Tajectory of subareas with 2 wanings.}
        \label{fig:deter_prodes_subarea_trajectory_2}
    \end{figure}
\end{frame}

\begin{frame}[allowframebreaks]
    \frametitle{DETER \& PRODES - Top 5 trajectories (2 warnings)}
    \begingroup\fontsize{7}{9}\selectfont

\begin{longtabu} to \linewidth {>{\raggedright}X>{\raggedright}X>{\raggedleft}X>{\raggedleft}X>{\raggedleft}X>{\raggedleft}X}
\toprule
position\_1 & position\_2 & area\_ha & n\_traj & p\_area & p\_traj\\
\midrule
\cellcolor{gray!6}{Cicatriz de queimada} & \cellcolor{gray!6}{P forest} & \cellcolor{gray!6}{1041281.5} & \cellcolor{gray!6}{12645} & \cellcolor{gray!6}{18.8} & \cellcolor{gray!6}{8.4}\\
\cmidrule{1-6}
Cicatriz de queimada & P deforestation & 927627.3 & 10848 & 16.8 & 7.2\\
\cmidrule{1-6}
\cellcolor{gray!6}{Desmatamento cr} & \cellcolor{gray!6}{P forest} & \cellcolor{gray!6}{838982.5} & \cellcolor{gray!6}{39589} & \cellcolor{gray!6}{15.2} & \cellcolor{gray!6}{26.3}\\
\cmidrule{1-6}
Desmatamento cr & P deforestation & 575218.9 & 29323 & 10.4 & 19.5\\
\cmidrule{1-6}
\cellcolor{gray!6}{P deforestation} & \cellcolor{gray!6}{Desmatamento cr} & \cellcolor{gray!6}{386346.7} & \cellcolor{gray!6}{21391} & \cellcolor{gray!6}{7.0} & \cellcolor{gray!6}{14.2}\\
\cmidrule{1-6}
Total & - & 5537714.8 & 150521 & 100.0 & 100.0\\
\bottomrule
\end{longtabu}
\endgroup{}
\end{frame}

\begin{frame}
    \frametitle{DETER \& PRODES subareas (3 warnings)}
    \begin{figure}[h] 
        \includegraphics[width=0.65\linewidth]
        {./figures/plot_deter_prodes_subarea_trajectory_3.png}
        \caption{Tajectory of subareas with 3 wanings.}
        \label{fig:deter_prodes_subarea_trajectory_3}
    \end{figure}
\end{frame}

\begin{frame}[allowframebreaks]
    \frametitle{DETER \& PRODES - Top 5 trajectories (3 warnings)}
    \begingroup\fontsize{7}{9}\selectfont

\begin{longtabu} to \linewidth {>{\raggedright}X>{\raggedright}X>{\raggedright}X>{\raggedleft}X>{\raggedleft}X}
\toprule
position\_1 & position\_2 & position\_3 & area\_ha & n\_traj\\
\midrule
\cellcolor{gray!6}{Cicatriz de queimada} & \cellcolor{gray!6}{Cicatriz de queimada} & \cellcolor{gray!6}{P forest} & \cellcolor{gray!6}{227564.0} & \cellcolor{gray!6}{4775}\\
\cmidrule{1-5}
Cicatriz de queimada & Cicatriz de queimada & P deforestation & 124744.1 & 2836\\
\cmidrule{1-5}
\cellcolor{gray!6}{Cicatriz de queimada} & \cellcolor{gray!6}{P deforestation} & \cellcolor{gray!6}{Cicatriz de queimada} & \cellcolor{gray!6}{115222.7} & \cellcolor{gray!6}{2983}\\
\cmidrule{1-5}
Cicatriz de queimada & Desmatamento cr & P forest & 30929.6 & 1888\\
\cmidrule{1-5}
\cellcolor{gray!6}{Desmatamento cr} & \cellcolor{gray!6}{Desmatamento cr} & \cellcolor{gray!6}{P forest} & \cellcolor{gray!6}{24851.9} & \cellcolor{gray!6}{2624}\\
\cmidrule{1-5}
Total & - & - & 523312.4 & 15106\\
\bottomrule
\end{longtabu}
\endgroup{}
\end{frame}

\begin{frame}
    \frametitle{DETER \& PRODES subareas (4 warnings)}
    \begin{figure}[h] 
        \includegraphics[width=0.65\linewidth]
        {./figures/plot_deter_prodes_subarea_trajectory_4.png}
        \caption{Tajectory of subareas with 4 wanings.}
        \label{fig:deter_prodes_subarea_trajectory_4}
    \end{figure}
\end{frame}

\begin{frame}[allowframebreaks]
    \frametitle{DETER \& PRODES - Top 5 trajectories (4 warnings)}
    \begingroup\fontsize{7}{9}\selectfont

\begin{longtabu} to \linewidth {>{\raggedright}X>{\raggedright}X>{\raggedright}X>{\raggedright}X>{\raggedleft}X>{\raggedleft}X>{\raggedleft}X>{\raggedleft}X}
\toprule
position\_1 & position\_2 & position\_3 & position\_4 & area\_ha & n\_traj & p\_area & p\_traj\\
\midrule
\cellcolor{gray!6}{Cicatriz de queimada} & \cellcolor{gray!6}{Cicatriz de queimada} & \cellcolor{gray!6}{Cicatriz de queimada} & \cellcolor{gray!6}{P forest} & \cellcolor{gray!6}{32093.7} & \cellcolor{gray!6}{524} & \cellcolor{gray!6}{22.5} & \cellcolor{gray!6}{11.8}\\
\cmidrule{1-8}
Cicatriz de queimada & Cicatriz de queimada & P deforestation & Cicatriz de queimada & 21441.9 & 387 & 15.0 & 8.7\\
\cmidrule{1-8}
\cellcolor{gray!6}{Cicatriz de queimada} & \cellcolor{gray!6}{Cicatriz de queimada} & \cellcolor{gray!6}{Cicatriz de queimada} & \cellcolor{gray!6}{P deforestation} & \cellcolor{gray!6}{10783.9} & \cellcolor{gray!6}{236} & \cellcolor{gray!6}{7.6} & \cellcolor{gray!6}{5.3}\\
\cmidrule{1-8}
Cicatriz de queimada & P deforestation & Cicatriz de queimada & Cicatriz de queimada & 5683.2 & 116 & 4.0 & 2.6\\
\cmidrule{1-8}
\cellcolor{gray!6}{Cicatriz de queimada} & \cellcolor{gray!6}{Cicatriz de queimada} & \cellcolor{gray!6}{Degradacao} & \cellcolor{gray!6}{P forest} & \cellcolor{gray!6}{5222.6} & \cellcolor{gray!6}{102} & \cellcolor{gray!6}{3.7} & \cellcolor{gray!6}{2.3}\\
\cmidrule{1-8}
Total & - & - & - & 142544.6 & 4451 & 100.0 & 100.0\\
\bottomrule
\end{longtabu}
\endgroup{}
\end{frame}

\begin{frame}
    \frametitle{DETER \& PRODES subareas (5 warnings)}
    \begin{figure}[h] 
        \includegraphics[width=0.65\linewidth]
        {./figures/plot_deter_prodes_subarea_trajectory_5.png}
        \caption{Tajectory of subareas with 5 wanings.}
        \label{fig:deter_prodes_subarea_trajectory_5}
    \end{figure}
\end{frame}

\begin{frame}[allowframebreaks]
    \frametitle{DETER \& PRODES - Top 5 trajectories (5 warnings)}
    \begingroup\fontsize{7}{9}\selectfont

\begin{longtabu} to \linewidth {>{\raggedright}X>{\raggedright}X>{\raggedright}X>{\raggedright}X>{\raggedright}X>{\raggedleft}X>{\raggedleft}X}
\toprule
position\_1 & position\_2 & position\_3 & position\_4 & position\_5 & area\_ha & n\_traj\\
\midrule
\cellcolor{gray!6}{Cicatriz de queimada} & \cellcolor{gray!6}{Cicatriz de queimada} & \cellcolor{gray!6}{Cicatriz de queimada} & \cellcolor{gray!6}{Cicatriz de queimada} & \cellcolor{gray!6}{P forest} & \cellcolor{gray!6}{1290.8} & \cellcolor{gray!6}{33}\\
\cmidrule{1-7}
Cicatriz de queimada & Cicatriz de queimada & Cicatriz de queimada & P deforestation & Cicatriz de queimada & 384.5 & 14\\
\cmidrule{1-7}
\cellcolor{gray!6}{Cicatriz de queimada} & \cellcolor{gray!6}{Cicatriz de queimada} & \cellcolor{gray!6}{Cicatriz de queimada} & \cellcolor{gray!6}{Cicatriz de queimada} & \cellcolor{gray!6}{P deforestation} & \cellcolor{gray!6}{364.5} & \cellcolor{gray!6}{8}\\
\cmidrule{1-7}
Cicatriz de queimada & P deforestation & Cicatriz de queimada & Cicatriz de queimada & Degradacao & 308.7 & 3\\
\cmidrule{1-7}
\cellcolor{gray!6}{Degradacao} & \cellcolor{gray!6}{Cs geometrico} & \cellcolor{gray!6}{Cs geometrico} & \cellcolor{gray!6}{Cicatriz de queimada} & \cellcolor{gray!6}{P forest} & \cellcolor{gray!6}{279.9} & \cellcolor{gray!6}{1}\\
\cmidrule{1-7}
Total & - & - & - & - & 2628.4 & 59\\
\bottomrule
\end{longtabu}
\endgroup{}
\end{frame}

\begin{frame}
    \frametitle{DETER \& PRODES subareas (6 warnings)}
    \begin{figure}[h] 
        \includegraphics[width=0.65\linewidth]
        {./figures/plot_deter_prodes_subarea_trajectory_6.png}
        \caption{Tajectory of subareas with 6 wanings.}
        \label{fig:deter_prodes_subarea_trajectory_6}
    \end{figure}
\end{frame}

\begin{frame}[allowframebreaks]
    \frametitle{DETER \& PRODES - Top 5 trajectories (6 warnings)}
    \begingroup\fontsize{7}{9}\selectfont

\begin{longtabu} to \linewidth {>{\raggedright}X>{\raggedright}X>{\raggedright}X>{\raggedright}X>{\raggedright}X>{\raggedright}X>{\raggedleft}X>{\raggedleft}X}
\toprule
position\_1 & position\_2 & position\_3 & position\_4 & position\_5 & position\_6 & area\_ha & n\_traj\\
\midrule
\cellcolor{gray!6}{Cicatriz de queimada} & \cellcolor{gray!6}{Cicatriz de queimada} & \cellcolor{gray!6}{Cicatriz de queimada} & \cellcolor{gray!6}{Cicatriz de queimada} & \cellcolor{gray!6}{Cicatriz de queimada} & \cellcolor{gray!6}{P forest 2021} & \cellcolor{gray!6}{187.4} & \cellcolor{gray!6}{4}\\
\cmidrule{1-8}
Degradacao & Cs geometrico & Cs geometrico & P desmatamento & Cs geometrico & Degradacao & 74.3 & 1\\
\cmidrule{1-8}
\cellcolor{gray!6}{Cicatriz de queimada} & \cellcolor{gray!6}{Cicatriz de queimada} & \cellcolor{gray!6}{P desmatamento} & \cellcolor{gray!6}{Cicatriz de queimada} & \cellcolor{gray!6}{Cicatriz de queimada} & \cellcolor{gray!6}{Cicatriz de queimada} & \cellcolor{gray!6}{55.7} & \cellcolor{gray!6}{2}\\
\cmidrule{1-8}
Cicatriz de queimada & Cicatriz de queimada & Cicatriz de queimada & Cicatriz de queimada & P desmatamento & Cicatriz de queimada & 30.4 & 3\\
\cmidrule{1-8}
\cellcolor{gray!6}{Cs geometrico} & \cellcolor{gray!6}{Degradacao} & \cellcolor{gray!6}{Cs geometrico} & \cellcolor{gray!6}{P desmatamento} & \cellcolor{gray!6}{Cs geometrico} & \cellcolor{gray!6}{Cs geometrico} & \cellcolor{gray!6}{21.9} & \cellcolor{gray!6}{1}\\
\cmidrule{1-8}
Cs geometrico & Degradacao & Cs geometrico & Cs geometrico & Cs geometrico & P forest 2021 & 11.0 & 2\\
\cmidrule{1-8}
\cellcolor{gray!6}{Degradacao} & \cellcolor{gray!6}{Cs geometrico} & \cellcolor{gray!6}{Cs geometrico} & \cellcolor{gray!6}{Cs geometrico} & \cellcolor{gray!6}{Degradacao} & \cellcolor{gray!6}{P forest 2021} & \cellcolor{gray!6}{6.3} & \cellcolor{gray!6}{1}\\
\cmidrule{1-8}
Cs geometrico & Cs geometrico & P desmatamento & Cs geometrico & Cs geometrico & Cs geometrico & 3.4 & 1\\
\cmidrule{1-8}
\cellcolor{gray!6}{Cs geometrico} & \cellcolor{gray!6}{Cs geometrico} & \cellcolor{gray!6}{Cs geometrico} & \cellcolor{gray!6}{Cs geometrico} & \cellcolor{gray!6}{P desmatamento} & \cellcolor{gray!6}{Cs geometrico} & \cellcolor{gray!6}{3.2} & \cellcolor{gray!6}{1}\\
\cmidrule{1-8}
Total & - & - & - & - & - & 393.5 & 16\\
\bottomrule
\end{longtabu}
\endgroup{}
\end{frame}

\begin{frame}[allowframebreaks]
    \frametitle{DETER \& PRODES - proximity in time}
    \begin{itemize}
        \item There is a potential DETER-PRODES overlap in our analysis.
        \item DETER warnings and PRODES deforestation polygons could refer to
            the same event.
        \item The next slide reports the DETER warnings closest in 
            time to PRODES polygons.
    \end{itemize}
\end{frame}

\begin{frame}[allowframebreaks]
    \frametitle{DETER \& PRODES - Top 5 proximity in time}
    \begingroup\fontsize{7}{9}\selectfont

\begin{longtabu} to \linewidth {>{\raggedright}X>{\raggedright}X>{\raggedleft}X>{\raggedleft}X>{\raggedleft}X>{\raggedleft}X>{\raggedleft}X>{\raggedleft}X}
\toprule
CLASSNAME & closest\_class & total\_ha & n & median\_days & median\_days\_abs & sd\_days & sd\_abs\\
\midrule
\cellcolor{gray!6}{P desmatamento} & \cellcolor{gray!6}{Cicatriz de queimada} & \cellcolor{gray!6}{1620833.2} & \cellcolor{gray!6}{26272} & \cellcolor{gray!6}{300} & \cellcolor{gray!6}{396} & \cellcolor{gray!6}{687.2} & \cellcolor{gray!6}{496.9}\\
P forest 2021 & Cicatriz de queimada & 1338484.8 & 19411 & 688 & 688 & 516.1 & 515.8\\
\cellcolor{gray!6}{P desmatamento} & \cellcolor{gray!6}{Desmatamento cr} & \cellcolor{gray!6}{1049533.4} & \cellcolor{gray!6}{57922} & \cellcolor{gray!6}{64} & \cellcolor{gray!6}{383} & \cellcolor{gray!6}{731.8} & \cellcolor{gray!6}{504.0}\\
P forest 2021 & Desmatamento cr & 939836.4 & 47172 & 649 & 649 & 488.8 & 488.4\\
\cellcolor{gray!6}{P desmatamento} & \cellcolor{gray!6}{Degradacao} & \cellcolor{gray!6}{391152.0} & \cellcolor{gray!6}{11266} & \cellcolor{gray!6}{353} & \cellcolor{gray!6}{478} & \cellcolor{gray!6}{731.9} & \cellcolor{gray!6}{511.0}\\
P forest 2021 & Degradacao & 303475.8 & 8198 & 1028 & 1028 & 540.6 & 540.4\\
\cellcolor{gray!6}{P desmatamento} & \cellcolor{gray!6}{Cs desordenado} & \cellcolor{gray!6}{218994.5} & \cellcolor{gray!6}{1861} & \cellcolor{gray!6}{11} & \cellcolor{gray!6}{364} & \cellcolor{gray!6}{746.3} & \cellcolor{gray!6}{531.8}\\
P forest 2021 & Cs desordenado & 194497.7 & 1526 & 401 & 401 & 530.4 & 530.0\\
\cellcolor{gray!6}{P desmatamento} & \cellcolor{gray!6}{Cs geometrico} & \cellcolor{gray!6}{168756.0} & \cellcolor{gray!6}{1658} & \cellcolor{gray!6}{44} & \cellcolor{gray!6}{348} & \cellcolor{gray!6}{636.2} & \cellcolor{gray!6}{464.5}\\
P forest 2021 & Cs geometrico & 130931.0 & 1155 & 379 & 379 & 473.8 & 473.6\\
\bottomrule
\end{longtabu}
\endgroup{}
\end{frame}



\subsection{Analysis 1}

\begin{frame}
    \frametitle{Analysis 1}
    \begin{itemize}
        \item Trajectories have one event each PRODES year. There were 
            70/517059 with more than one.
        \item Trajectories with mining in them were excluded.
        \item Trajectories end as soon as they reach deforestation.
        \item Trajectories include at least one PRODES event.
    \end{itemize}
\end{frame}

\begin{frame}
    \frametitle{Analysis 1 (2 warnings) }
    \begin{figure}[h] 
    \includegraphics[width=0.65\textwidth]{./figures/an1_plot_deter_prodes_subarea_trajectory_2.png}
    \end{figure}
\end{frame}

\begin{frame}
    \frametitle{Analysis 1 (3 warnings) }
    \begin{figure}[h] 
    \includegraphics[width=0.65\textwidth]{./figures/an1_plot_deter_prodes_subarea_trajectory_3.png}
    \end{figure}
\end{frame}

\begin{frame}
    \frametitle{Analysis 1 (4 warnings) }
    \begin{figure}[h] 
    \includegraphics[width=0.65\textwidth]{./figures/an1_plot_deter_prodes_subarea_trajectory_4.png}
    \end{figure}
\end{frame}

\begin{frame}
    \frametitle{Analysis 1 (5 warnings) }
    \begin{figure}[h] 
    \includegraphics[width=0.65\textwidth]{./figures/an1_plot_deter_prodes_subarea_trajectory_5.png}
    \end{figure}
\end{frame}

\begin{frame}
    \frametitle{Analysis 1 (6 warnings) }
    \begin{figure}[h] 
    \includegraphics[width=0.65\textwidth]{./figures/an1_plot_deter_prodes_subarea_trajectory_6.png}
    \end{figure}
\end{frame}


\subsection{Analysis 2}

\begin{frame}
    \frametitle{Analysis 2}
    \begin{itemize}
        \item Same as Analysis 1, but only using DETER's burn scars.
    \end{itemize}
\end{frame}

\begin{frame}
    \frametitle{Analysis 2 (2 warnings) }
    \begin{figure}[h] 
    \includegraphics[width=0.65\textwidth]{./figures/an2_plot_deter_prodes_subarea_trajectory_2.png}
    \end{figure}
\end{frame}

\begin{frame}
    \frametitle{Analysis 2 (3 warnings) }
    \begin{figure}[h] 
    \includegraphics[width=0.65\textwidth]{./figures/an2_plot_deter_prodes_subarea_trajectory_3.png}
    \end{figure}
\end{frame}

\begin{frame}
    \frametitle{Analysis 2 (4 warnings) }
    \begin{figure}[h] 
    \includegraphics[width=0.65\textwidth]{./figures/an2_plot_deter_prodes_subarea_trajectory_4.png}
    \end{figure}
\end{frame}

\begin{frame}
    \frametitle{Analysis 2 (5 warnings) }
    \begin{figure}[h] 
    \includegraphics[width=0.65\textwidth]{./figures/an2_plot_deter_prodes_subarea_trajectory_5.png}
    \end{figure}
\end{frame}

\begin{frame}
    \frametitle{Analysis 2 (6 warnings) }
    \begin{figure}[h] 
    \includegraphics[width=0.65\textwidth]{./figures/an2_plot_deter_prodes_subarea_trajectory_6.png}
    \end{figure}
\end{frame}



\section{Closing}

\begin{frame}
    \frametitle{Final remarks}
    \begin{itemize}
        \item The analysis of DETER warning subareas along time could improve 
            the characterization of forest degradation along time.
        \item Potential applications of our work are:
            \begin{itemize}
                \item Improve estimation of emissions of greenhouse gases, i.e.
                    our data could help avoiding double counting.
                \item Identify spatio-temporal areas which could help training 
                    Machine-Learning algorithms for automatic indentification 
                    of forest degradation.
            \end{itemize}
        \item Code available at 
            \url{https://github.com/albhasan/treesburnareas}
    \end{itemize}
\end{frame}

\begin{frame}[allowframebreaks]
    \frametitle{References}
    \bibliographystyle{amsalpha}
    \bibliography{07_SBSR.bib}
\end{frame}

\end{document}
