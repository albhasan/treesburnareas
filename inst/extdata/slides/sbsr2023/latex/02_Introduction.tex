\section{Introduction}

% Background, known information.
The Amazon forest plays an important role in the current climate crisis.
Besides hosting a large number of species and regulating the water and carbon 
cycles, the forest works as a large carbon storage and it is frequently cited  
as one of the tipping points, which ---if mishandled--- could potentially cause 
an abrupt and irreversible change in the climate 
system~\cite{rockstrom2009,armstrong2022}.

% Knowledge gap, unknown information.
Advances in the areas of Ecology, Remote Sensing, and Computer Science have
fostered regional, continental and even global deforestation
monitoring systems (e.g. PRODES, Global Forest Watch).
However, detecting forest degradation is more challenging than detecting
deforestation~\cite{lambin1999,mitchell2017}.

% Hypothesis, question, purpose statement.
Due to the importance of what is at stake, we seek new ways to improve 
degradation characterization which could alleviate the difficulties 
associated to detection of forest degradation.
For this reason, in this manuscript, we explore the possibilities of using 
DETER warnings for understanding forest degradation.
DETER constantly issues deforestation warnings on the Brazilian Amazon forest, 
and these warnings capture forest degradation at different stages of
development.
This information has many potential applications, such as for example, 
providing quality training data for Machine Learning algorithms able to process 
massive amounts of satellite imagery.

% Approach, plan of attack, proposed solution.
To test the feasibility of our approach, we processed 5 years of DETER 
warnings in the municipality of \textit{São Félix do Xingu}, estimating the 
areas and number of days between warnings on the same area.
