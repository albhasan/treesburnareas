\section{Discussion}

Our results show that the number of subareas with more than one DETER warning
is low compared to the total number of warnings and there are but a few of 
subareas with more than three warnings.
They also show that most of successive warnings of the same subarea are at most 
four years apart, two years from the first to the second, and one year from 
there. 
We found that DETER warnings provide between two and four warnings to 
characterize degradation processes in one of the most deforested municipalities 
in the Amazon. 

The number of available subareas seems small for training Machine Learning 
algorithms, specially those based on Deep Learning that are well-known
for requiring large amounts of training data.
However, we expect these numbers increase by extending our analysis to the 
whole area covered by the Brazilian Amazon since 2016.
In addition, we think our results foster new analysis in areas different from 
Computer Science.
Despite this fact, our results are important as they explore a potential new
application of the already useful and openly available DETER data.

Our results rely on the assumption that \textit{São Félix do Xingu} is 
representative of degradation in the Amazon.
They also rely on the accuracy of DETER warning polygons and the assumption
that subareas larger than 3 ha correspond to actual degradation warnings 
instead of drawing inaccuracies on a computer screen.

