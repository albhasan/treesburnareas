\begin{table}[h] % use [t] here to force table to top
    \centering
    \begin{tabular}{|c|c|r|r|}
        \hline
        \textbf{N. Warnings} & \textbf{Type} & 
        \textbf{N. Subareas} & \textbf{Total area}  \\
        \hline
1 & 6.25 ha    &  749  &   3468.9 \\ 
1 & 10 ha      & 2535  &  20277.6 \\
1 & 25 ha      & 4002  &  63098.2 \\
1 & 50 ha      & 1682  &  58496.4 \\
1 & 100 ha     &  847  &  58490.3 \\
1 & 250 ha     &  459  &  69741.0 \\
1 & 500 ha     &  133  &  43890.2 \\
1 & 1000 ha    &   54  &  36388.0 \\
1 & > 1000 ha &   42  & 108701.0  \\
        \hline
2 & 6.25 ha    &  617  &   2754.4 \\
2 & 10 ha      &  542  &   4300.3 \\
2 & 25 ha      &  829  &  13016.6 \\
2 & 50 ha      &  356  &  12588.7 \\
2 & 100 ha     &  200  &  14101.6 \\
2 & 250 ha     &  111  &  16949.5 \\
2 & 500 ha     &   25  &   8804.5 \\
2 & 1000 ha    &   12  &   8361.2 \\
2 & > 1000 ha &    3  &   6444.0  \\
        \hline
3 & 6.25 ha    &   70  &    331.8 \\
3 & 10 ha      &   48  &    383.2 \\
3 & 25 ha      &   62  &    943.4 \\
3 & 50 ha      &   15  &    528.5 \\
3 & 100 ha     &    5  &    342.0 \\
3 & 250 ha     &    2  &    261.5 \\
        \hline
4 & 6.25 ha    &    5  &     24.4 \\
4 & 10 ha      &    1  &      8.4 \\
4 & 25 ha      &    2  &     31.4 \\
        \hline
    \end{tabular}
    \caption{DETER warning subareas by number of wanings, type, number of 
    subareas, and total area. The number and total area decreases as the number
    of warnings increase.}
    \label{tab:warnings_subareas_by_number_area}
\end{table}
